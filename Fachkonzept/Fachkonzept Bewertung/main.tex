\documentclass[12pt, a4paper]{article}
\usepackage[titles]{tocloft}
\usepackage{xpatch}
\usepackage{etoolbox}

\usepackage[utf8]{inputenc}
\usepackage[ngerman]{babel}
\usepackage{hyphenat} % manuelle Trennung z. B.  Ma-Risk -> MaRisk
\usepackage[T1]{fontenc}
\usepackage{graphicx}
\usepackage{float}
\usepackage[hyphens]{url}
\usepackage{geometry}
\usepackage{calc}	
\usepackage{fancyhdr}
\usepackage{xcolor}
\usepackage{setspace}
% deaktiviere bunte Boxen, vergebe pdf meta informationen
\usepackage[hidelinks, pdfauthor={Markus Bilz, Christian Fix},
pdftitle={Fachkonzept für die Bewertung von Wertpapieren},
pdfkeywords={Anleihebewertung, Aktienbewertung}
]{hyperref}
% \usepackage[raggedright]{titlesec} % Vorgabe für Titelformatierung
\usepackage{caption}
\usepackage{subcaption}
\usepackage{acronym}
\usepackage{float}
\usepackage{enumitem}
\usepackage{booktabs}
\usepackage{tikz}
\usepackage{csquotes} % Behebt Fehler bei Zitaten in Babel
\usepackage{layout} 
\usepackage[all]{nowidow} % vermeidet Witten und Waisen
\usepackage{amsmath}
\usepackage[ngerman, num]{isodate} % wird für Datumsformatierung gebraucht
\usepackage{fncychap}
\usepackage[hang,flushmargin]{footmisc} % Formatierung der Fußnoten
\usepackage{lmodern} 
\usepackage[backend=biber,style=apa,citestyle=apa]{biblatex} % apa-stil für bibliography in biblatex
\DeclareLanguageMapping{german}{german-apa}
\usepackage[toc,section=section]{glossaries} % packet für Glossar ggf. auch für Abkürzungsverzeichnis
\usepackage{booktabs, tabularx, threeparttable}
\usepackage{ragged2e, array}
\usepackage{listings}% Aufzählungen

\usepackage{pgfplots}% pgf plots 
\pgfplotsset{compat=1.15}
\usepgfplotslibrary{statistics}
\usepgfplotslibrary{dateplot}

\geometry{a4paper,
    left=35mm,
    right=25mm,
    top= 25mm, 
    bottom=20mm,
    headsep=12.5mm,
    headheight=12.5mm,
    footskip=12.5mm,
    voffset = 0pt,
    hoffset = 0pt,
}

% Absatzeinstellungen
\setlength{\parindent}{0em}
\setlength{\parskip}{6pt}
\linespread{1.3} % Einfacher Zeilenabstand

% wichtig für Bilder
\graphicspath{{img/}}


% Kopf- und Fußzeilen einbinden
\pagestyle{fancy}
\fancyhf{}
\fancyfoot[R]{\thepage}
\renewcommand{\headrulewidth}{0.4pt} %obere Trennlinie
\renewcommand{\footrulewidth}{0.4pt} %untere Trennlinie
\newcommand{\sectionbreak}{\clearpage} % section auf neuer Seite

% Erhöhe Zeilenabstand bei Tabellen
\renewcommand{\arraystretch}{1.3}

% Einstellungen für underfull und overfull badness
\tolerance 1414
\hbadness 1414
\emergencystretch 1.5em
\widowpenalty=10000
\vfuzz=20pt 
\hfuzz=20pt

\title{Fachkonzept für die Bewertung von Finanzanlagen und die Verbuchung von Kapitalerträgen}
\author{Markus Bilz\thanks{markus.bilz@student.dhbw-karlsuhe.de}, Christian Fix\thanks{christian.fix@student.dhbw-karlsuhe.de}}

\addbibresource{bib.bib}

\begin{document}

\maketitle

\section{Überblick*}
Im Rahmen des MWI-Projektes soll die bestehende Planspielsoftware in der DHBW Karlsruhe um eine zusätzliche autarke Wertpapierkomponente mit dem Namen \textit{Anika} erweitert werden.
Um sicherzustellen, dass diese Wertpapiere möglichst realistisch und fair bewertet werden können, beschreibt dieses Fachkonzept, wie deren Bewertung in dieser Software durchgeführt werden soll.

Die Software \textit{Anika} unterstützt dabei den Handel folgender Finanzanlagen:
\begin{enumerate}
	\item Aktien
	\item Floating Rate Notes (ETF)
	\item Exchange Traded Funds
	\item Festgeld
\end{enumerate}

Weil \textit{Anika} kein dediziertes Handelssystem implementiert, das die Handelsaktivitäten einzelner Marktteilnehmer in der Kursbildung von Wertpapieren berücksichtigt, werden anwendungsweit ausschließlich rechnerische Kurswerte verwendet. 
Deshalb erfolgt sowohl die Bewertung als auch der Handel von Finanzanlagen zu einem rechnerischen Kurs, der auf Grundlage finanzmathematischer Modelle ermittelt wird.
Insofern ist die Modellierung des rechnerischen Werts einer Finanzanlage eine zentrale Funktionalität der Software.

%Nachfolgende Kapitel gehen detailliert auf die Bewertung obiger Anlagearten ein. Zunächst wird jedoch der Zeitpunkt und das Vorgehen hinsichtlich der Bewertung von Finanzanlagen thematisiert.

\section{Anlage und Pflege von Finanzanlagen $\ast$}
\label{sec:anlage_und_pflege_der_wertpapiere}

Bevor eine Bewertung der Wertpapiere erfolgen kann, müssen die handelbaren Wertpapiere einschließlich der für die Bewertung notwendigen Geschäftsdaten angelegt sein. 
Im Folgenden wird beschrieben, wie die Finanzanlagen angelegt und gepflegt werden:
\begin{itemize}
	\item Der ETF und das Festgeld wird automatisch bei der Anlage eines Spiels erstellt.
	\item Aktien und Floating Rate Notes können von den Planspielunternehmen emittiert werden, indem der Spielleiter dies bei der Anlage des Spiels manuell durchführt. Die Software ermöglicht dabei, dass je Planspielunternehmen $\geq 0$ Aktien und Floating Rate Notes emittiert werden können. 
\end{itemize}

Die Pflege der für die Bewertung notwendigen Daten wie beispielsweise die Erfassung des Kapitalmarktzinssatzes, des Risikoaufschlages oder des Aktienkurses erfolgt dabei einmalig vor dem Start einer Planspielperiode durch den Spielleiter.

\section{Zeitpunkt der Bewertung und Verbuchung}
\label{sec:zeitpunkt_und_durchfuehrung_der_bewertung_buchung}

Dieses Kapitel beschreibt den Zeitpunkt der Bewertung und der Verbuchung von Kapitelerträgen.

Das Planspiel \textit{TOPSIM} nutzt ein  

Die rechnerischen Kurswerte gelten damit jeweils für die Folgeperiode.


Da die Dauer einer Periode umfasst einen Tag. 

\begin{figure}[htb]
	\centering
	\begin{tikzpicture} 
		\draw[thick, ->] (0,0) -- (12cm,0);
		\foreach \x in {2,4,6,8,10}
		\draw (\x cm,3pt) -- (\x cm,-3pt);
		\draw[thick] (4,0) node[below=3pt,thick] {$31.12.$} node[above=3pt] {};
		\draw[thick] (6,0) node[below=3pt, thick] {$t_h$} node[above=3pt] {};
		\draw[thick] (10,0) node[below=3pt] {$T$} node[above=3pt] {};
		\draw [black, thick ,decorate,decoration={brace,amplitude=5pt}] (4,0.5)  -- (8,0.5) 
			   node [black,midway,above=4pt,font=\scriptsize] {Gültigkeit};
		\draw [ black, thick,decorate,decoration={brace,amplitude=5pt},yshift=-11pt] (10,-0.5) -- (8,-0.5)
			   node [black,midway,below=4pt,font=\scriptsize] {$T-t_h$};
		\end{tikzpicture}
	\caption{Bewertungszeitpunkt}
	\label{img:zeitstrahl_pewertung}
	(Eigene Darstellung)
\end{figure}

Damit lässt sich zusammenfassen, dass Aktienkurse und rechnerische Anleihekurse -- festgestellt am Jahresultimo -- die für die Folgeperiode relevanten Kurse für die Bewertung und den Handel darstellen. Die Verbuchung der Kapitalerträge erfolgt jeweils am Jahresultimo der Periode.

Die Bewertung der Finanzanlagen kann in einer beliebigen Reihenfolge efolgen. Ausschließlich für die Bewertung des ETFs bestehen temporale Abhängigkeiten zu anderen Anlagen. Kapitel \ref{sec:bewertung_eines_exchange_traded_funds} thematisiert dies detailliert.

\section{Bewertung von Finanzanlagen}
\label{sec:bewertung_von_finanzanlagen}

Nachfolgende Unterkapitel beschreiben die Bewertung der in der Software \textit{Anika} handelbaren Finanzanlagen.

\section{Bewertung von Aktien $\ast$}
\label{sec:bewertung_von_aktien}
Die Unternehmen des Planspiels firmieren als Aktiengesellschaft.
Ein rechnerischer Aktienkurs wird daher durch die Anwendung \textit{TOPSIM} auf Grundlage des Eigenkapitals und des Jahresüberschusses der vergangenen Periode nebst anderen Einflussfaktoren berechnet (\dots). 
Eine Gewichtung der Faktoren mit Einfluss auf den Aktienkurs ist in der Anwendung \textit{TOPSIM} konfigurierbar. 

Aktienkurse der Planspielunternehmen werden \textit{ex post} je Planspielunternehmen und Periode ermittelt. Die durch die Software \textit{TOPSIM} ermittelten Aktienkurse sind um Dividendenausschüttungen bereingt. Kapitalmaßnahmen mit Einfluss auf den Aktienkurs erfolgen darüber hinaus nicht, womit eine unmittelbare Verwendung in der Software \textit{Anika} möglich ist. Nachfolgend wird beschrieben, wie eine Erfassung der Aktienkurse erfolgen soll.

Gemäß Kapitel \ref{sec:zeitpunkt_und_durchfuehrung_der_bewertung_buchung} ist der Aktienkurs der Vorperiode der Bewertungskurs der Folgeperiode.
Zugleich ist ein Kauf und Verkauf zu diesem Kurs möglich -- bleiben Ordergebühren und eine Geld-/ und Briefspanne außer Acht. 

% TODO: Ausformulieren. Pflege von Kursen siehe Kapitel 2. Ordergebühren in %. Zuordnung sinnvoll zu Handel von Wertpapieren?

\subsection{Bewertung von Floating Rate Notes $\ast$}
\label{sec:bewertung_von_floating_rate_notes}
Floating Rate Notes sind Anleihen mit einem über die Laufzeit veränderlichen Zinskupon \autocite[][373]{fabozzi_handbook_2005}. Der Zinskupon setzt sich dabei aus einem aus einem Referenzzins und einen vom Emittenten abhängigen Aufschlag zusammen \autocite[][374]{fabozzi_handbook_2005}.

Anleihen sind kein Bestandteil der Anwendung \textit{TOPSIM}, insofern ist eine Bewertung durch die Anwendung \textit{Anika} notwendig.
Hierbei ist insbesondere eine Bewertung der An

Für die Implementierung wird eine Laufzeit von zehn Jahren, beginnend in Periode Null angenommen.
Die Floating Rate Note verfügt über keinen Cap, Floor oder Collar, der die Höhe des Zinskupons beschränkt.
Kündigungsrechte des Emittenten / Gläubigers bestehen nicht.

Da Zeitpunkt der Zinszahlung und Kauf- / und Zeitpunkt der Anleihe übereinstimmen, sind keine Stückzinsen zu berücksichtigen.

Die Bewertung orientert sich an \autocite[][]{veronesi_fixed_2010}.

Alternativ ist auch \autocite[][]{fabozzi_handbook_2005} möglich.

Der Wert eines Floaters mit einem Spread von $s=0$ entspricht der der Wert des Floater dem Nennwert des Floaters ohne Zinskupons \autocite[][S.~52~f.]{veronesi_fixed_2010}. Dies ist auf \glqq Zahlungsstromeffekte\grqq~und \glqq Diskontierungseffekte\grqq~zurückzuführen, die sich gegenseitig ausgleichen \autocite[][S.~54]{veronesi_fixed_2010}. Höhere Zinszahlungen -- resultierend aus einer Zinserhöhung -- werden durch einen höheren Diskontierungssätze kompensiert.

Es handelt sich dabei um einen Spezialfall, der die Anforderungen von \textit{Anika} nicht vollständig abdeckt, da zwar eine Bepreisung zum Zinszahlungstermin erfolgt, der Spread aber auch andere Werte annehmen kann.

Für einen Floater mit einem Spread lässt sich der Cashflow in eine fixe und eine veränderliche Komponente aufspalten.

Damit teilt sich die Bewertung in die Bewetung einer Floating Rate Note mit einem Spread von Null und mehreren festen Zinszahlungen in Höhe des Spreads auf.

% https://www.finpricing.com/lib/FiFrn.pdf

\subsection{Bewertung eines Exchange Traded Funds $\dagger$}
\label{sec:bewertung_eines_exchange_traded_funds}
Bei {ETFs} handelt es sich um eine börsengehandelte Variante des Investmentfonds, die es Anlegern ermöglicht, Portfolios, welche einen Index replizieren, zu handeln \autocite[][S.~103]{bodie_investments_2018}. 

Für die Bewertung des ETFs wird dabei eine vollständige Replizierung des Indizes unterstellt.
Der \textit{Tracking Difference}\footnote{Der \textit{Tracking Difference} bezeichnet die Renditedifferenz zwischen dem ETF und dem abgebildeten Index.} und die \textit{Total Expense Ratio}\footnote{Die \textit{Total Expense Ration} bezeichnet die Gesamtkostenquote des Fonds. Hierunter fallen beispielsweise Kosten zur Erfüllung regulatorischer Anforderungen.} werden dabei mit Null angenommen.
Dies bedeutet, dass der ETF der Entwicklung des zugrundeliegenden Indizes 1:1 folgt.

% TODO: ETF bezieht sich hier auf Aktien der Planspielunternehmen
% TODO: Thesaurierender ETF.

Für die rechnerische Bewertung ist deshalb die Berechnung des \textit{GMAX} Indizes durch die Anwendung erforderlich. So ist die Berechnung des Indizes erst nach Pflege aller im Index enthaltenen Aktien möglich.

\subsection{Berechnung des {GMAX} Index $\dagger$}

Es bestehen temporale Abhängigkeiten bei der Berechnung des Index. Die 

% https://en.wikipedia.org/wiki/Stock_market_index
% \section{Ausblick}
% Das Grundsätzlich ist auch eine Bewert
% Weiterhin ist . Die propagierten Modelle sind dabei geeignet, um eine 

% Kurz beschreiben

\section{Bewertung von Festgeld $\ast$}
\label{sec:bewertung_von_festgeldern}

Ein Festgeld ist eine Variante der Termineinlage, dessen Kapital für eine vertraglich vereinbarte Anlagedauer fixiert ist.

Die Ausgestaltung von Festgeldern in der Software \textit{Anika} folgt dabei nachfolgenden Konventionen. Die Anlagedauer beträgt eine Periode.
Die Verzinsung entspricht dem Kapitalmarktzins für die Dauer einer Periode. Es erfolgt eine automatische Prolongation des Festgelds um eine Periode. Eine jederzeitige Verfügung den Teilnehmer ist im vollen Umfang oder teilweise ohne Vorfälligkeitsentschädigung möglich.
Weiterhin sind Aufstockungen des Festgeld allzeit möglich.
% Eine Anlage in Festgelder ist für die Teilnehmer des Planspiels optional.

Festgelder werden zum Kapitalsaldo bewertet. 

\section{Ermittlung von Kapitalerträgen $\dagger$}
\label{sec:ermittlung_von_wertpapierertraegen}

\subsection{Ausschüttungen aus Aktie $\ast$}
\label{sec:ausschuettung_aus_aktie}
% TODO: Gibt es eine klarere Bezeichnung als um Dividendenausschüttungen bereinigt?
Da der verwendete Aktienkurs um Dividendenausschüttungen bereinigt ist und deshalb gezahlte Dividenden im Kurs berücksichtigt, erfolgt keine zusätzliche Ausschüttung von Dividenden (siehe Kapitel \ref{sec:bewertung_von_aktien}).

\subsection{Ausschüttungen aus ETF $\ast$}
\label{sec:ausschuettungen_aus_etf}
Ausschüttungen aus dem ETF erfolgen nicht, da der {ETF} als thesaurierender {ETF} aufgelegt ist (siehe Kapitel \ref{sec:bewertung_eines_exchange_traded_funds}).

\subsection{Zinserträge auf Festgelder $\ast$}
\label{sec:zinsertraege_auf_festgelder}
Die Dauer der Priode beträgt 1 Tag (siehe Kapitel \ref{sec:zeitpunkt_und_durchfuehrung_der_bewertung_buchung}).
Maßgeblich für die Bewertung ist der Tagesendsaldo 


\subsection{Zinserträge auf Floating Rate Notes $\ast$}
\label{sec:zinsertraege_auf_floating_rate_notes}

\printbibliography[title={Literatur}]
\end{document}
