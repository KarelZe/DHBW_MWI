\documentclass[12pt, a4paper]{article}
\usepackage[titles]{tocloft}
\usepackage{xpatch}
\usepackage{etoolbox}

\usepackage[utf8]{inputenc}
\usepackage[ngerman]{babel}
\usepackage{hyphenat} % manuelle Trennung z. B.  Ma-Risk -> MaRisk
\usepackage[T1]{fontenc}
\usepackage{graphicx}
\usepackage[hyphens]{url}
\usepackage{geometry}
\usepackage{calc}	
\usepackage{fancyhdr}
\usepackage{xcolor}
\usepackage{setspace}
% deaktiviere bunte Boxen, vergebe pdf meta informationen
\usepackage[hidelinks, pdfauthor={Markus Bilz, Christian Fix},
pdftitle={Fachkonzept für die Bewertung von Wertpapieren},
pdfkeywords={Anleihebewertung, Aktienbewertung}
]{hyperref}
% \usepackage[raggedright]{titlesec} % Vorgabe für Titelformatierung
\usepackage{caption}
\usepackage{subcaption}
\usepackage{acronym}
\usepackage{float}
\usepackage{enumitem}
\usepackage{booktabs}
\usepackage{tikz}
\usepackage{csquotes} % Behebt Fehler bei Zitaten in Babel
\usepackage{layout} 
\usepackage[all]{nowidow} % vermeidet Witten und Waisen
\usepackage{amsmath}
\usepackage[ngerman, num]{isodate} % wird für Datumsformatierung gebraucht
\usepackage{fncychap}
\usepackage[hang,flushmargin]{footmisc} % Formatierung der Fußnoten
\usepackage{lmodern} 
\usepackage[backend=biber,style=apa,citestyle=apa]{biblatex} % apa-stil für bibliography in biblatex
\DeclareLanguageMapping{german}{german-apa}
\usepackage[toc,section=section]{glossaries} % packet für Glossar ggf. auch für Abkürzungsverzeichnis
\usepackage{booktabs, tabularx, threeparttable}
\usepackage{ragged2e, array}
\usepackage{listings}% Aufzählungen

\usepackage{amsthm}
\theoremstyle{plain}
\newtheorem{definition}{Definition}
\newtheorem{example}{Beispiel}

\usepackage{pgfplots}% pgf plots 
\pgfplotsset{compat=1.15}
\usepgfplotslibrary{statistics}
\usepgfplotslibrary{dateplot}

\geometry{a4paper,
    left=35mm,
    right=25mm,
    top= 25mm, 
    bottom=20mm,
    headsep=12.5mm,
    headheight=12.5mm,
    footskip=12.5mm,
    voffset = 0pt,
    hoffset = 0pt,
}

% Absatzeinstellungen
\setlength{\parindent}{0em}
\setlength{\parskip}{6pt}
\linespread{1.3} % Einfacher Zeilenabstand

% wichtig für Bilder
\graphicspath{{img/}}


% Kopf- und Fußzeilen einbinden
\pagestyle{fancy}
\fancyhf{}
\fancyfoot[R]{\thepage}
\renewcommand{\headrulewidth}{0.4pt} %obere Trennlinie
\renewcommand{\footrulewidth}{0.4pt} %untere Trennlinie
\newcommand{\sectionbreak}{\clearpage} % section auf neuer Seite

% Erhöhe Zeilenabstand bei Tabellen
\renewcommand{\arraystretch}{1.3}

% Einstellungen für underfull und overfull badness
\tolerance 1414
\hbadness 1414
\emergencystretch 1.5em
\widowpenalty=10000
\vfuzz=20pt 
\hfuzz=20pt

\title{Fachkonzept für die Bewertung von Finanzanlagen und die Verbuchung von Kapitalerträgen in der Software Anika}
\author{Markus Bilz\thanks{markus.bilz@student.dhbw-karlsuhe.de}, Christian Fix\thanks{christian.fix@student.dhbw-karlsuhe.de}}

\addbibresource{bib.bib}

\newacronym{ETF}{ETF}{Exchange Traded Fund}
\newacronym{FRN}{FRN}{Floating Rate Note}
\newacronym{LIBOR}{LIBOR}{London Interbank Offered Rate}
\newacronym{GMAX}{GMAX}{General Management Aktienindex}

\makeatletter
\let\@fnsymbol\@arabic
\makeatother

\begin{document}

\maketitle

\section{Überblick}
Im Rahmen des MWI-Projektes soll die bereits in der DHBW Karlsruhe eingesetzte Planspielsoftware \textit{TOPSIM} um eine zusätzliche autarke Wertpapierkomponente mit dem Namen \textit{Anika} erweitert werden, um den Teilnehmern die Möglichkeit zu bieten, mit einem fiktiven Kapital Finanzanlagen zu erwerben.

Die Software \textit{Anika} soll dabei den Handel folgender Finanzanlagen unterstützen:
\begin{enumerate}
	\item Aktien
	\item \gls{FRN}
	\item \gls{ETF}
	\item Festgeld
\end{enumerate}

Um sicherzustellen, dass diese Finanzanlagen möglichst realistisch und fair bewertet werden, wurde dieses Fachkonzept erstellt, das definiert, wie deren Bewertung und Ausschüttungen in dieser Software durchgeführt werden soll.
Weil \textit{Anika} kein dediziertes Handelssystem implementiert, das die Handelsaktivitäten einzelner Marktteilnehmer in der Kursbildung von Wertpapieren berücksichtigt, werden dafür ausschließlich rechnerische Kurswerte verwendet, der auf der Grundlage finanzmathematischer Modelle ermittelt wird.

%Nachfolgende Kapitel gehen detailliert auf die Bewertung obiger Anlagearten ein. Zunächst wird jedoch der Zeitpunkt und das Vorgehen hinsichtlich der Bewertung von Finanzanlagen thematisiert.

\section{Anlage und Pflege von Finanzanlagen}
\label{sec:anlage_und_pflege_der_wertpapiere}

Bevor eine Bewertung der Finanzanlagen erfolgen kann, müssen sowohl diese als auch die die Bewertung notwendigen Geschäftsdaten angelegt werden. 
Im Folgenden wird deshalb beschrieben, wie die Finanzanlagen angelegt und gepflegt werden sollen:
\begin{itemize}
	\item Bei der Anlage eines Spiels durch den Seminarleiter soll automatisch ein \gls{ETF} und ein Festgeld angelegt werden, sodass die Teilnehmer diese Finanzanlagen handeln können.
	\item Im Rahmen der Initialisierung eines Spiels können zusätzlich sowohl Aktien als auch \glspl{FRN} der Planspielunternehmen manuell emittiert werden.  
\end{itemize}

Die Pflege der Daten wie des Kapitalmarktzinssatzes, des unternehmensabhängigen Risikoaufschlages oder des Aktienkurses, die für die Bewertung der Finanzanlagen benötigt werden, erfolgt dabei einmalig vor dem Start einer Planspielperiode durch den Spielleiter.

\section{Zeitpunkt der Bewertung und Verbuchung}
\label{sec:zeitpunkt_und_durchfuehrung_der_bewertung_buchung}

%Dieses Kapitel beschreibt den Zeitpunkt der Bewertung und der Verbuchung von Kapitelerträgen.

Die Software \textit{TOPSIM} unterteilt ein Planspiel in $n$ Perioden $P$.
Eine feingranulare Unterteilung einer Periode ist nicht möglich, weshalb die Dauer einer Periode mit einer Zeiteinheit angenommen wird.
Daraus folgt, dass der Periodenbeginn von $P_1$ dem Ultimo der Vorperiode $P_0$ entspricht.

Demnach ergibt sich folgender Zusammmenhang:

\begin{figure}[htb]
	\centering
	\begin{tikzpicture} 
		\draw[thick, ->] (0,0) -- (12cm,0);
		\foreach \x in {0,4,8}
		\draw (\x cm,3pt) -- (\x cm,-3pt);
		\draw[thick] (4,0) node[below=3pt,thick] {Ultimo $P_0$} node[above=3pt] {};
		\draw[thick] (8,0) node[below=3pt, thick] {Ultimo $P_1$} node[above=3pt] {};
		\draw [thick ] (4,0.5) node [above=4pt,font=\scriptsize, align=left] {
		Verbuchung $P_0$\\ \parindent=1em \indent Bewertung $P_1$\\ \parindent=2em \indent Kauf/Verkauf $P_1$};
		\draw [ black, thick,decorate,decoration={brace,amplitude=5pt},yshift=-11pt] (8,-0.5) -- (4,-0.5)
			   node [black,midway,below=4pt,font=\scriptsize] {Dauer der Verzinsung};
		\draw [thick ] (8,0.5) node [above=4pt,font=\scriptsize, align=left] {Verbuchung $P_1$\\ \parindent=1em \indent Bewertung $P_2$\\ \parindent=2em \indent Kauf/Verkauf $P_2$};
	\end{tikzpicture}
	\caption{Bewertungs-/ Buchungszeitpunkt}
	\label{img:zeitstrahl_pewertung}
	(Eigene Darstellung)
\end{figure}

Damit lässt sich zusammenfassen, dass Aktienkurse und rechnerische Anleihekurse, die am Ultimo der Vorperiode festgestellt werden, die für die Folgeperiode relevanten Kurse für die Bewertung und den Handel darstellen.
Die Bewertung der Finanzanlagen kann dabei grundsätzlich in einer beliebigen Reihenfolge erfolgen.
Lediglich für die Bewertung des \glspl{ETF} bestehen temporale Abhängigkeiten zu anderen Anlagen. Kapitel \ref{sec:bewertung_eines_exchange_traded_funds} thematisiert dies detailliert.

Um sicherzustellen, dass die Finanzanlagen immer zu einem fairen Kurs gehandelt werden, können diese erst gehandelt werden, nachdem sie bewertet wurden. 
Die Verbuchung der Kapitalerträge erfolgt jeweils am Ultimo der Periode nach Durchführung aller Kauf- und Verkaufbuchungen.
%Die Dauer der Verzinsung beträgt damit ein Tag. %ToDo Markus: warum schreibst du das? hätte ich jetzt weggelassen

\section{Bewertung von Finanzanlagen}
\label{sec:bewertung_von_finanzanlagen}
Im Folgenden wird beschrieben, wie die Finanzanlagen in der Software \textit{Anika} bewertet werden sollen.

\subsection{Bewertung von Aktien}
\label{sec:bewertung_von_aktien}
Die Planspielunternehmen firmieren als Aktiengesellschaft, deren Aktien von den Teilnehmern gehandelt werden können.
Der rechnerische Kurs dieser Aktien wird von der Planspielsoftware \textit{TOPSIM} auf der Basis einiger Einflussfaktoren wie beispielsweise dem Eigenkapital oder dem Jahresüberschuss der vergangenen Periode berechnet und dem Seminarleiter in einer Übersicht dargestellt.
%Die Methodik, wie sich dieser Aktienkurs berechnet wird, ist dabei grundsätzlich konfigurierbar.
Dieser Aktienkurs beinhaltet die vergangenen Dividendenauszahlungen. Auch zukünftige Dividenden werden nicht ausgeschüttet, sondern wirken sich positiv auf den Kurs aus. Dies führt dazu, dass die Software \textit{Anika} die Dividendenauszahlungen nict gesondert berücksichtigen muss.
%Die Aktienkurse der Planspielunternehmen werden \textit{ex post} Periode ermittelt und dem Seminarleiter in einer Übersicht dargestellt. 

Gemäß Kapitel \ref{sec:zeitpunkt_und_durchfuehrung_der_bewertung_buchung} ist der Aktienkurs der Vorperiode der Bewertungskurs der Folgeperiode.
Bei dem Handel mit Aktien wird neben deren Kurswert eine vom Seminarleiter eingestellte Ordergebühr\footnote{Diese Ordergebühr wird in Prozent angegeben.} fällig. Eine in der Realität oft auftretende Brief-Geld-Spanne existiert hingegen nicht.

\subsection{Bewertung von Floating Rate Notes}
\label{sec:bewertung_von_floating_rate_notes}
\glspl{FRN} sind Anleihen mit einem über die Laufzeit veränderlichen Zinskupon \autocite[][373]{fabozzi_handbook_2005}. Der Zinskupon setzt sich dabei aus einem aus einem Referenzzins und einen von der Bonität des Emittenten abhängigen Zinsaufschlages zusammen \autocite[][374]{fabozzi_handbook_2005}.

Anleihen sind kein Bestandteil der Anwendung \textit{TOPSIM}, insofern ist eine Bewertung durch die Anwendung \textit{Anika} notwendig.
Hierbei ist insbesondere eine Bewertung der An %ToDo Markus

Für die Implementierung wird eine Laufzeit von zehn Perioden, beginnend in Periode $P_0$ angenommen.
Die \gls{FRN} verfügt über keinen Cap, Floor oder Collar, der die Höhe des Zinskupons beschränkt.
Kündigungsrechte des Emittenten / Gläubigers bestehen nicht.

Da Zeitpunkt der Zinszahlung und Kauf- / und Zeitpunkt der Anleihe übereinstimmen, sind keine Stückzinsen zu berücksichtigen.

Die Bewertung orientert sich an \autocite[][]{veronesi_fixed_2010}.

% Floater passen sich zwar der finanziellen Situation der Firma an, nicht aber der 
% Kursverlauf aufnehmen.

Alternativ ist auch \autocite[][]{fabozzi_handbook_2005} möglich.

Der Wert einer \gls{FRN} mit einem Spread von $s=0$ entspricht der der Wert der \gls{FRN} dem Nennwert der \gls{FRN} ohne Zinskupons \autocite[][S.~52~f.]{veronesi_fixed_2010}. Dies ist auf \glqq Zahlungsstromeffekte\grqq~und \glqq Diskontierungseffekte\grqq~zurückzuführen, die sich gegenseitig ausgleichen \autocite[][S.~54]{veronesi_fixed_2010}. Höhere Zinszahlungen -- resultierend aus einer Zinserhöhung -- werden durch einen höheren Diskontierungssätze kompensiert.

Es handelt sich dabei um einen Spezialfall, der die Anforderungen von \textit{Anika} nicht vollständig abdeckt, da zwar eine Bepreisung zum Zinszahlungstermin erfolgt, der Spread aber auch andere Werte annehmen kann.

Für einer \gls{FRN} mit einem Spread lässt sich der Cashflow in eine fixe und eine veränderliche Komponente aufspalten.

Damit teilt sich die Bewertung in die Bewetung einer \gls{FRN} mit einem Spread von Null und mehreren festen Zinszahlungen in Höhe des Spreads auf.

Für die Bewertung von \gls{FRN} gelten folgende Prämissen:
\begin{itemize}
	\item Die Bewertung erfolgt am Kupon\dots
	\item Restlaufzeit...
\end{itemize}

% https://www.finpricing.com/lib/FiFrn.pdf

% Berechnung Discount Factor

Damit ist eine Bepreisung \dots

Das ist, Autoren wie \textcite{veronesi_fixed_2010} propagieren ein identisches Vorgehen.

\begin{align*}
	P_{t+T}^{s}&=\left(B_{t+T}^{s}-B_{t+T}^{0}\right)-100(1+c)\left(1+t R_{0}\right)^{-1}\\
	s&= Emissionsspread
\end{align*}

\textcite[][]{alexander_market_2008} zerlegt dabei die fixierten Zahlungen aus dem Spread $s$ und den veränderlichen Zahlungn aus dem Referenzzins. Die Festzins

Nachfolgendes Beispiel ist \textcite[][32]{alexander_market_2008} entlehnt.

Bewertet wird ein \gls{FRN} mit jährlichen Zinskupons, die sich aus Referenzzins \gls{LIBOR} plus 60 Basispunkte zusammensetzen. Die Bewertung erfolgt an einem Zinszahlungstermin. Der \gls{LIBOR} zum Bewertungszeitpunkt beträgt 5~\%. Die Abzinsungssätze für 2, 3, 4 Jahre 4,85~\%, 4,65~\% beziehungsweise 4,5~\%.
% Die Diskontierung erfolgt mit dem Libor

Da eine Zerlegung in feste Anleihe und einen Variablen Zahlungsstrom erfolgt, ist zunächst die Festzinsanleihe (erster Term) zu bewerten.

Die Bewertung des variablen Zahlungsstroms vereinfacht sich, da eine Bepreisung zum Zinszahlungsdatum erfolgt. Der zweite Term kann mit 100 angenommen. Damit ergibt sich:

\begin{align*}
	P_{t+T}^{s}&=\left(B_{t+T}^{s}-B_{t+T}^{0}\right)-100(1+c)\left(1+t R_{0}\right)^{-1}\\
	B_{t+T}^{s}&=\frac{0.60}{1.050}+\frac{0.60}{1.0485^2}+\frac{0.60}{1.0465^3}+\frac{100.60}{1.045^4}=86.00\\
	B_{t+T}^{0}&=\frac{100}{1.045^4}=83.856\\
	P_{t+T}^{s}&= (86.00-83.856) + 100 = 102.144\\
\end{align*}

% TODO: Wie lässt sich geänderte Bonität berücksichtigen? Höhere Abzinsungsfaktoren?

% TODO: Jave EE nennt es static spread der wird aufgeschlagwn
\begin{align*}
	P=\sum_{i=1}^{n} \frac{C_{i}}{\left(1+r_{s}+r(i)\right)^{i}}+\frac{100}{\left(1+r_{s}+r(n)^{n}\right.}
\end{align*}

Legt man Arbitragefreiheit zugrunde, dann entspricht 
% Mittels eines No-Arbitrage Arguments lässt sich zeigen, dass ein Floater im Fixingtermin immer zu pari notiert. Eine Floating Rate Note lässt sich synthetisch erzeugen durch eine rollierende (revolvierende) Geldanlage in einer Fixed Rate Note. Diese müssen dann denselben Preis haben.

% Achtung!: Dies gilt nur für den Fall, dass der Bonitätsspread dem Bonitätsrisiko entspricht (sem=qm).

Dies stellt eine vereinfachende Annahme, 

\begin{definition}
	\label{def:floater}
	\dots
\end{definition}

% Siehe Tuckman S. 542 f für Überlegungen zu Credit Spread.

\subsection{Bewertung eines Exchange Traded Funds}
\label{sec:bewertung_eines_exchange_traded_funds}

Bei \glspl{ETF} handelt es sich um eine börsengehandelte Variante des Investmentfonds, die es Anlegern ermöglicht, Portfolios, die einen Index replizieren, zu handeln \autocite[][S.~103]{bodie_investments_2018}. Bei dem zugrundeliegenden Index kann es sich dabei beispielsweise um einen Aktien- oder Anleihenindex handeln, deren Wertentwicklung abgebildet wird.

Die Berechnung des Index setzt Konventionen zur Gewichtung der Anlagen voraus. Ein Überblick über Ansätze zur Gewichtung wird in \textcite[][S.~44~ff.]{bodie_investments_2018} gegeben, wohin gegen sich dieses Fachkonzept auf verwendete Ansätze beschränkt.

Die Software \textit{Anika} bietet jedem Teilnehmer die Möglichkeit, einen ETF zu handeln, der die Wertentwicklung des Index \gls{GMAX} repliziert (siehe nachfolgende Definition \ref{def:gmax}).

\begin{definition}
	\label{def:gmax}
	Der \gls{GMAX} ist ein preisgewichteter (\textit{price weighted}) Aktienindex der Planspielunternehmen, bei dem alle Aktien gleichgewichtet sind.
\end{definition}

Die Berechnung des \gls{GMAX} als preisgewichteter Aktienindex erlaubt damit eine einfache und nachvollziehbare Berechnung. Zugleich stellt es einen Ansatz dar, der praktische Bedeutung für Indizes wie dem usprünglichen \textit{Dow Jones Industrial Average} hat, weshalb dieses Verfahren für die Software \textit{Anika} ausgewählt wurde. Dieser Ansatz hat jedoch den Nachteil, dass hoch bewertete Aktien einen größeren Einfluss auf die Indexentwicklung nehmen.

Die Berechnung des \gls{GMAX} wird an Beispiel~\ref{ex:gmax}\footnote{Das Beispiel ist von \textcite[][S.~44]{bodie_investments_2018} adaptiert.} für ein Spiel mit zwei Planspielunternehmen erläutert.

\begin{example}
\label{ex:gmax}
Legt man einen Aktienkurs für Unternehmen A $25$ ($P_0$) und $30$ in ($P_1$) und für Unternehmen B von $100$ ($P_0$) und $90$ in ($P_1$) zugrunde, dann kann der Indexstand zum Bewertungszeitpunkt wie folgt ermittelt werden:\\
Indexstand \gls{GMAX} ($P_0$) = $\frac{(25 + 100)}{2} = 62.5$\\
Indexstand \gls{GMAX} ($P_1$) = $\frac{(30 + 90)}{2} = 60.$

Die prozentuale Veränderung von $P_0$ auf $P_1$ ergibt damit aus $-\frac{2.5}{62.5} = -4.0~\%$.
\end{example}

 Anhand des \gls{GMAX} wird ein \gls{ETF} konstruiert. Die Ausgestaltung des \gls{ETF} in der Software \textit{Anika} kann der nachfolgenden Definition~\ref{def:etf} entnommen werden.
\begin{definition}
	\label{def:etf}
	Der \gls{ETF} trackt den Index \gls{GMAX} durch vollständige Replizierung. Die \textit{Tracking Difference}\footnote{Der \textit{Tracking Difference} bezeichnet die Renditedifferenz zwischen dem \gls{ETF} und dem abgebildeten Index.} und die \textit{Total Expense Ratio}\footnote{Die \textit{Total Expense Ration} bezeichnet die Gesamtkostenquote des Fonds. Hierunter fallen beispielsweise Kosten zur Erfüllung regulatorischer Anforderungen.} wird eleminiert, wodurch die Wertentwicklung des \gls{ETF} die des \gls{GMAX} 1:1 abbildet.
\end{definition}

Eine Bewertung des \gls{ETF} ist deshalb erst dann möglich, wenn alle im \gls{GMAX} enthaltenen Aktienkurse vorliegen. Aufgrund der Ausgestaltung gemäß Beispiel~\ref{def:etf} entspricht der Kurs des \glspl{ETF} dem Preis des \gls{GMAX} in Euro. 

\subsection{Bewertung von Festgeld}
\label{sec:bewertung_von_festgeldern}

Als ein Festgeld wird eine Variante der Termineinlage bezeichnet, dessen Kapital für eine vertraglich vereinbarte Anlagedauer fixiert ist.

Die Ausgestaltung von Festgeldern in der Software \textit{Anika} unterscheidet sich dabei in Teilen von den üblichen am Markt befindlichen Festgeldern. Im Definition \ref{def:festgeld} werden deshalb die Konditionen des in \textit{Anika} verwendeten Festgelds dargestellt.

\begin{definition}
	\label{def:festgeld}
	Das Festgeld wird mit dem periodebanhängigen Kapitalmarktzinssatz verzinst und hat eine Laufzeit von einer Periode mit automatischer Prolongation um eine weitere Periode. Teilverfügungen und vollständige Verfügungen und Aufstockungen sind jederzeit durch den Teilnehmer ohne Vorfälligkeitsentschädigung möglich. 
\end{definition}

Weil die Festgelder mit dem jeweiligen Kapitalmarktzinssatz verzinst werden, werden sie mit dem jeweiligen Kapitalsaldo bewertet, da sie dadurch effektiv einer risikolosen Floating Rate Note gleichzusetzen sind. Bei dem Kauf bzw. Verkauf des Festgeldes wird jedoch im Gegensatz zu den anderen Finanzanlagen keine Ordergebühr fällig.

\section{Ermittlung der Kapitalerträge}
\label{sec:ermittlung_von_wertpapierertraegen}

Nachfolgende Kapitel beschreiben die Ermittlung der Kapitalerträge für die in \textit{Anika} auftretenden Finanzanlagen.

\subsection{Ausschüttungen aus Aktien und ETFs}
\label{sec:ausschuettung_aus_aktie}
Wie bereits in Kapitel \ref{sec:bewertung_von_aktien} beschrieben wurde, beinhaltet der Aktienkurs der Planspielunternehmen bereits die ausgeschütteten Dividendenauszahlungen. Aus diesem Grund soll keine separate Dividendenausschüttung erfolgen. Dies gilt auch für \glspl{ETF}.

\subsection{Zinserträge auf Festgelder}
\label{sec:zinsertraege_auf_festgelder}
Festgelder werden, wie bereits in Kapitel \ref{sec:bewertung_von_festgeldern} beschrieben wurde, mit dem jeweiligen Kapitalmarktzinssatz verzinst. Diese Zinszahlung wird am Ende der jeweiligen Periode auf das Zahlungsmittelkonto des Teilnehmers gutgeschrieben.

\subsection{Zinserträge auf Floating Rate Notes}
\label{sec:zinsertraege_auf_floating_rate_notes}

Die Inhaber von \gls{FRN} erhalten gemäß Definition \ref{def:floater} Zinsausschüttungen.
Wegen der periodischen Zinsfixings und Ausschüttung wird der Zinsbetrag nach nachfolgender Definition \ref{def:zins_floater}\footnote{Diese Definition ist aus \textcite[][S.~52]{veronesi_fixed_2010} adaptiert.} bestimmt.

\begin{definition}
	\label{def:zins_floater}
\begin{align*}
	c_t &= 100 * (r_{t-1} + s)\\
	c_t &= \text{Zinszahlung per $t$}\\
	s &= \text{Spread bei Emission}\\
	r_t &= \text{Referenzzins per $t$}
\end{align*}
$c_t$ wird dem Zahlungsmittelkonto gutgeschrieben.
\end{definition}

Die Berechnung der Zinserträge der \gls{FRN} erfolgt damit bezogen auf den Nennbetrag $(= 100)$ unter Verwendung des Kapitalmarktzinssatzes und des Spreads. Dabei ist der zeitliche Verzug zwischen dem Zinsfixing und dem Termin der Kuponzahlung erkennbar.

\begin{example}
	Eine \gls{FRN} muss bepreist werden. Der Kapitalmarktzins zum zurückliegenden Fixingtermin betrug 2~\%. Der \textit{Spread} bei Emission der \gls{FRN} beträgt 1~\%. 
	
	Die Kuponzahlung $c_t$ ergibt sich wiefolgt:

	$c_t= 100 (0.01 + 0.02) = 3 $ Euro
\end{example}
\clearpage
\printbibliography[title={Literatur}]
\end{document}
