\documentclass[12pt, a4paper]{article}
\usepackage[titles]{tocloft}
\usepackage{xpatch}
\usepackage{etoolbox}

\usepackage[utf8]{inputenc}
\usepackage[ngerman]{babel}
\usepackage{hyphenat} % manuelle Trennung z. B.  Ma-Risk -> MaRisk
\usepackage[T1]{fontenc}
\usepackage{graphicx}
\usepackage[hyphens]{url}
\usepackage{geometry}
\usepackage{calc}	
\usepackage{fancyhdr}
\usepackage{xcolor}
\usepackage{setspace}
% deaktiviere bunte Boxen, vergebe pdf meta informationen
\usepackage[hidelinks, pdfauthor={Markus Bilz, Christian Fix},
pdftitle={Fachkonzept für die Bewertung von Wertpapieren},
pdfkeywords={Anleihebewertung, Aktienbewertung}
]{hyperref}
% \usepackage[raggedright]{titlesec} % Vorgabe für Titelformatierung
\usepackage{caption}
\usepackage{subcaption}
\usepackage{acronym}
\usepackage{float}
\usepackage{enumitem}
\usepackage{booktabs}
\usepackage{tikz}
\usepackage{csquotes} % Behebt Fehler bei Zitaten in Babel
\usepackage{layout} 
\usepackage[all]{nowidow} % vermeidet Witten und Waisen
\usepackage{amsmath}
\usepackage[ngerman, num]{isodate} % wird für Datumsformatierung gebraucht
\usepackage{fncychap}
\usepackage[hang,flushmargin]{footmisc} % Formatierung der Fußnoten
\usepackage{lmodern} 
\usepackage[backend=biber,style=apa,citestyle=apa]{biblatex} % apa-stil für bibliography in biblatex
\DeclareLanguageMapping{german}{german-apa}
\usepackage[toc,section=section]{glossaries} % packet für Glossar ggf. auch für Abkürzungsverzeichnis
\usepackage{booktabs, tabularx, threeparttable}
\usepackage{ragged2e, array}
\usepackage{listings}% Aufzählungen

\usepackage{amsthm}
\theoremstyle{plain}
\newtheorem{definition}{Definition}
\newtheorem{example}{Beispiel}

\usepackage{pgfplots}% pgf plots 
\pgfplotsset{compat=1.15}
\usepgfplotslibrary{statistics}
\usepgfplotslibrary{dateplot}

\geometry{a4paper,
    left=35mm,
    right=25mm,
    top= 25mm, 
    bottom=20mm,
    headsep=12.5mm,
    headheight=12.5mm,
    footskip=12.5mm,
    voffset = 0pt,
    hoffset = 0pt,
}

% Absatzeinstellungen
\setlength{\parindent}{0em}
\setlength{\parskip}{6pt}
\linespread{1.3} % Einfacher Zeilenabstand

% wichtig für Bilder
\graphicspath{{img/}}


% Kopf- und Fußzeilen einbinden
\pagestyle{fancy}
\fancyhf{}
\fancyfoot[R]{\thepage}
\renewcommand{\headrulewidth}{0.4pt} %obere Trennlinie
\renewcommand{\footrulewidth}{0.4pt} %untere Trennlinie
\newcommand{\sectionbreak}{\clearpage} % section auf neuer Seite

% Erhöhe Zeilenabstand bei Tabellen
\renewcommand{\arraystretch}{1.3}

% Einstellungen für underfull und overfull badness
\tolerance 1414
\hbadness 1414
\emergencystretch 1.5em
\widowpenalty=10000
\vfuzz=20pt 
\hfuzz=20pt

\title{Fachkonzept für die Bewertung von Finanzanlagen und die Verbuchung von Kapitalerträgen in der Software Anika}
\author{Markus Bilz\thanks{markus.bilz@student.dhbw-karlsruhe.de}, Christian Fix\thanks{christian.fix@student.dhbw-karlsruhe.de}}

\addbibresource{bib.bib}

\newacronym{ETF}{ETF}{Exchange Traded Fund}
\newacronym{FRN}{FRN}{Floating Rate Note}
\newacronym{LIBOR}{LIBOR}{London Interbank Offered Rate}
\newacronym{GMAX}{GMAX}{General Management Aktienindex}

\makeatletter
\let\@fnsymbol\@arabic
\makeatother

\begin{document}

\maketitle

\section{Überblick}
Im Rahmen des MWI-Projektes soll die bereits in der DHBW Karlsruhe eingesetzte Planspielsoftware \textit{TOPSIM} um eine zusätzliche autarke Wertpapierkomponente mit dem Namen \textit{Anika} erweitert werden, um den Teilnehmern die Möglichkeit zu bieten, mit einem fiktiven Kapital Finanzanlagen zu erwerben.

Die Software \textit{Anika} soll dabei den Handel folgender Finanzanlagen unterstützen:
\begin{enumerate}
	\item Aktien
	\item \gls{FRN}
	\item \gls{ETF}
	\item Festgeld
\end{enumerate}

Um sicherzustellen, dass diese Finanzanlagen möglichst realistisch und fair bewertet werden, wurde dieses Fachkonzept erstellt, das definiert, wie deren Bewertung und Ausschüttungen in dieser Software durchgeführt werden soll.
Weil \textit{Anika} kein dediziertes Handelssystem implementiert, das die Handelsaktivitäten einzelner Marktteilnehmer in der Kursbildung von Wertpapieren berücksichtigt, werden dafür ausschließlich rechnerische Kurswerte verwendet, die auf der Grundlage finanzmathematischer Modelle ermittelt wird.

%Nachfolgende Kapitel gehen detailliert auf die Bewertung obiger Anlagearten ein. Zunächst wird jedoch der Zeitpunkt und das Vorgehen hinsichtlich der Bewertung von Finanzanlagen thematisiert.

\section{Anlage und Pflege der Finanzanlagen}
\label{sec:anlage_und_pflege_der_wertpapiere}

Bevor eine Bewertung der Finanzanlagen erfolgen kann, müssen sowohl diese als auch die für die Bewertung notwendigen Geschäftsdaten angelegt werden. 
Im Folgenden wird deshalb beschrieben, wie die Finanzanlagen angelegt und gepflegt werden sollen:
\begin{itemize}
	\item Bei der Anlage eines Spiels durch den Spielleiter soll automatisch ein \gls{ETF} und ein Festgeld angelegt werden, sodass die Teilnehmer diese Finanzanlagen jederzeit handeln können.
	\item Im Rahmen der Initialisierung eines Spiels können zusätzlich sowohl Aktien als auch \glspl{FRN} der Planspielunternehmen durch den Spielleiter emittiert werden.  
\end{itemize}

Die Pflege der Geschäftsdaten wie des Kapitalmarktzinssatzes, des unternehmensabhängigen Risikoaufschlages oder des Aktienkurses, die für die Bewertung der Finanzanlagen benötigt werden, erfolgt dabei einmalig vor dem Start einer Planspielperiode durch den Spielleiter.

\section{Zeitpunkt der Bewertung und Verbuchung}
\label{sec:zeitpunkt_und_durchfuehrung_der_bewertung_buchung}

%Dieses Kapitel beschreibt den Zeitpunkt der Bewertung und der Verbuchung von Kapitelerträgen.

Die Software \textit{TOPSIM} unterteilt ein Planspiel in $n$ Perioden $P$.
Eine feingranulare Unterteilung einer Periode ist nicht möglich, weshalb die Dauer einer Periode mit einer Zeiteinheit angenommen wird.
Daraus folgt, dass der Periodenbeginn von $P_1$ dem Ultimo der Vorperiode $P_0$ entspricht.

Demnach ergibt sich folgender Zusammmenhang:

\begin{figure}[htb]
	\centering
	\begin{tikzpicture} 
		\draw[thick, ->] (0,0) -- (12cm,0);
		\foreach \x in {0,4,8}
		\draw (\x cm,3pt) -- (\x cm,-3pt);
		\draw[thick] (4,0) node[below=3pt,thick] {Ultimo $P_0$} node[above=3pt] {};
		\draw[thick] (8,0) node[below=3pt, thick] {Ultimo $P_1$} node[above=3pt] {};
		\draw [thick ] (4,0.5) node [above=4pt,font=\scriptsize, align=left] {
		Verbuchung $P_0$\\ \parindent=1em \indent Bewertung $P_1$\\ \parindent=2em \indent Kauf/Verkauf $P_1$};
		\draw [ black, thick,decorate,decoration={brace,amplitude=5pt},yshift=-11pt] (8,-0.5) -- (4,-0.5)
			   node [black,midway,below=4pt,font=\scriptsize] {Dauer der Verzinsung};
		\draw [thick ] (8,0.5) node [above=4pt,font=\scriptsize, align=left] {Verbuchung $P_1$\\ \parindent=1em \indent Bewertung $P_2$\\ \parindent=2em \indent Kauf/Verkauf $P_2$};
	\end{tikzpicture}
	\caption{Bewertungs-/ Buchungszeitpunkt}
	\label{img:zeitstrahl_pewertung}
	(Eigene Darstellung)
\end{figure}

Damit lässt sich zusammenfassen, dass Aktienkurse und rechnerische Anleihekurse, die am Ultimo der Vorperiode festgestellt werden, die für die Folgeperiode relevanten Kurse für die Bewertung und den Handel darstellen.
Die Bewertung der Finanzanlagen kann dabei grundsätzlich in einer beliebigen Reihenfolge erfolgen.
Lediglich für die Bewertung des \glspl{ETF} bestehen temporale Abhängigkeiten zu anderen Anlagen. Kapitel \ref{sec:bewertung_eines_exchange_traded_funds} thematisiert dies detailliert.

Um sicherzustellen, dass die Finanzanlagen immer zu einem fairen Kurs gehandelt werden, können diese erst gehandelt werden, nachdem sie bewertet wurden. 
Die Verbuchung der Kapitalerträge erfolgt jeweils am Ultimo der Periode nach Durchführung aller Kauf- und Verkaufbuchungen.
%Die Dauer der Verzinsung beträgt damit ein Tag. %ToDo Markus: warum schreibst du das? hätte ich jetzt weggelassen

\section{Bewertung von Finanzanlagen}
\label{sec:bewertung_von_finanzanlagen}
Im Folgenden wird beschrieben, wie die Finanzanlagen in der Software \textit{Anika} bewertet werden sollen.

\subsection{Bewertung von Aktien}
\label{sec:bewertung_von_aktien}
Die Planspielunternehmen firmieren als Aktiengesellschaft, deren Aktien von den Teilnehmern gehandelt werden können.
Der rechnerische Kurs dieser Aktien wird von der Planspielsoftware \textit{TOPSIM} auf der Basis einiger Einflussfaktoren wie beispielsweise dem Eigenkapital oder dem Jahresüberschuss der vergangenen Periode berechnet und dem Spielleiter in einer Übersicht dargestellt.
Dieser Aktienkurs beinhaltet die vergangenen Dividendenauszahlungen. Auch zukünftige Dividenden werden nicht ausgeschüttet, sondern wirken sich positiv auf den Kurs aus. Dies führt dazu, dass die Software \textit{Anika} die Dividendenauszahlungen nicht gesondert berücksichtigen muss.

Im Insolvenzfall eines Planspielunternehmens ist der Kurs mit 0 zu pflegen.

%Die Aktienkurse der Planspielunternehmen werden \textit{ex post} Periode ermittelt und dem Spielleiter in einer Übersicht dargestellt. 

Gemäß Kapitel \ref{sec:zeitpunkt_und_durchfuehrung_der_bewertung_buchung} ist der Aktienkurs der Vorperiode der Bewertungskurs der Folgeperiode.
Bei dem Handel mit Aktien wird neben deren Kurswert eine vom Spielleiter eingestellte Ordergebühr\footnote{Diese Ordergebühr wird in Prozent angegeben.} fällig. Eine in der Realität oft auftretende Brief-Geld-Spanne existiert hingegen nicht.

Die Anwendung \textit{Anika} weißt Aktienkurse in \textit{Stücknotierung} aus.

\subsection{Bewertung von Floating Rate Notes}
\label{sec:bewertung_von_floating_rate_notes}
\glspl{FRN} sind Anleihen mit einem über die Laufzeit veränderlichen Zinskupon \autocite[][373]{fabozzi_handbook_2005}. Der Zinskupon setzt sich dabei aus einem Referenzzins und einen von der Bonität des Emittenten abhängigen Zinsaufschlags zusammen \autocite[][374]{fabozzi_handbook_2005}. In diesem Zusammenhang ist wichtig festzuhalten, dass sich zwar der Referenzzins über die Laufzeit verändert, der Aufschlag auf den Referenzzins bleibt jedoch konstant.

Weil Anleihen kein Bestandteil der Anwendung \textit{TOPSIM} sind, ist eine Bewertung durch die Anwendung \textit{Anika} notwendig. \gls{FRN} werden gemäß Definition \ref{def:floater} in der Anwendung \textit{Anika} aufgelegt.

\begin{definition}
	\label{def:floater}
	\gls{FRN} werden in Periode $P_0$ mit einer Laufzeit von zehn Perioden emittiert. Die Laufzeit eines Zinskupons beträgt eine Periode. \textit{Payment date} und \textit{reset date} entsprechen dem Periodenultimo. Der Zinskupon setzt sich aus dem veränderlichen Kapitalmarktzins und einem emissionsindividuellen Spread zusammen.
\end{definition}

In diesem Zusammenhang soll auf einige Besonderheiten aus Definition \ref{def:floater} eingegangen werden:
\begin{itemize}
	\item Die Laufzeit der Anleihe soll nicht konfigurierbar sein. Sie wird deshalb pauschal mit zehn Perioden angenommen. Eine fest vereinbarte Laufzeit ist notwendig, da andernfalls die Standard-Bewertungsmodelle für Floater nicht anwendbar sind. 
	\item Der Zinskupon referenziert den Kapitalmarktzins, da er der einzige Referenzzins in der Software ist. Der bei Emission erfasste Spreadaufschlag auf den Referenzzins bleibt für die Dauer des Spiels konstant.
	\item Die Einschränkung, dass Zinszahlungen einmalig zum Periodenultimo erfolgen, hebt die Notwendigkeit mehrerer Zinsbuchungen je Periode auf.
\end{itemize}

Die Bewertung der \gls{FRN} erfordert ein Bewertungsmodell. Ein Modell für die Bewertung der \gls{FRN} ist in \textcite[][S.~31~f.]{alexander_market_2008} dokumentiert. 
Es ermöglicht die Berpeisung von \glspl{FRN} mit einem Spread $\neq0$ zu einem beliebigen Bewertungstichtag.
Ausfallrisiken lassen sich damit nur vereinfacht abbilden.
Eine mögliche Alternative hierzu wird in \textcite[][S.~65~f.]{schonbucher_credit_2003} präsentiert. Allerdings verfügt sie über eine höhere Komplexität und erfordert weitere Annahmen hinsichtlich der Bonität. Aus diesem Grund wird für den Ansatz von \textcite[][S.~31~f.]{alexander_market_2008} optiert.

Angenommen, ein \gls{FRN} wird zum Zeitpunkt $t = 0$ bepreist, der nächste Zinskupon wurde in Höhe von $c$ am zurückliegenden \textit{payment date} fixiert und wird zu $t$ bezahlt. Überdies stehen $T$ Zinszahlungen bis zur Fälligkeit aus. Dann gehen dem Anleger zu den Zeitpunkten $t+1, t+2, \ldots, t+T$ Zinszahlungen und zusätzlich im Zeitpunkt $t+T$ die Rückzahlung der Anleihe $N$ zu.

Der Zinskupon setzt sich dabei aus einem Refrenzzins und einem Spread bei Emission vereinbarten Spread zusammen. Der einperiodische Referenzzins zum Zeitpunkt $t$ wird als $R_t$ bezeichnet. Diskontierung erfolgt ebenfalls zum Referenzzins (ggf. erhöht um einen Bonitätsaufschlag). Der Spread wird als $s$ bezeichnet. Damit entspricht zum Zeitpunkt $t+n$ die Höhe einer Zinszahlung $100(R_{t+n-1}+s)$. 

Der Preis der \gls{FRN} $P_{s}^{m}$ mit Spread $s$ und Fälligkeit $m$ ergibt sich damit als:
\begin{align*}
	P_{t+T}^{s}&=(\underbrace{B_{t+T}^{s}}_\text{Standardanleihe}-\underbrace{B_{t+T}^{0}}_\text{Nullkuponanleihe})+\underbrace{100(1+c - s)(1+t R_{0})^{-1}}_\text{Variabler Cashflow}\\
\end{align*}
% Obacht, Formel im Original ist falsch ist Teil des Errata http://carolalexander.org/publish/download/MRA_III_Revisions_1.pdf
Wie oben ersichtlich, zerlegt \textcite[][31]{alexander_market_2008} die \gls{FRN} in einen fest- und einen variabelverzinslichen Zahlungsstrom. Der festverzinsliche Zahlungsstrom umfasst dabei die diskontierten Zahlungen auf den Spread $s$ zu den Zeitpunkt $t, t+1, \ldots, t+T$. Sie bildet den festverzinslichen Zahlungsstrom in Form einer Standardanleihe $B_{t+T}^{s}$ ab, reduziert den Barwert aber um den Barwert einer Nullkuponanleihe $B_{t+T}^{0}$, da der festverzinsliche Zahlungsstrom keine Tilgung bei Fälligkeit enthält. Die Bewertung der Nullkuponanleihe und der Festzinsanleihe erfolgt durch Diskontierung, wie in \textcite[][11]{alexander_market_2008} beschrieben. Weiterhin diskontiert sie die zukünftigen, variabel verzinslichen Zinszahlungen an den Zeitpunkten $t+1, t+2, \ldots, t+T$ und die Tilgung in $t+T$  zum Zeitpunkt ab. 

Da in der Anwendung \textit{Anika} ausschließlich der Kapitalmarktzinssatz für die aktuelle Periode vorliegt, erfolgt die Diskontierung mittels Kapitalmarktzinssatzes erhöht um einen emittentenindividuellen Bonitätsaufschlag, der sich aus dem Rating je Periode ableitet. Die Diskontierungssätze sind dabei über alle Laufzeiten gleich.

Die Anwendung \textit{Anika} lässt darüber hinaus zu, den rechnerischen Kurs einer \gls{FRN} durch einen manuellen Kurs zu überschreiben. Dies ist insbesondere im Insolvenzfall eines Unternehmens sinnvoll.

Die Anwendung \textit{Anika} weißt Anleihekurse in \textit{Prozentnotierung} aus.

Nachfolgend wird die exemplarische Bewertung einer \gls{FRN} demonstriert\footnote{Nachfolgendes Beispiel ist aus \textcite[][32]{alexander_market_2008} adaptiert.}. 

\begin{example}
	Bewertet wird ein \gls{FRN} mit einem jährlichen Zinskupon, der sich aus dem Kapitalmarktzins plus 60 Basispunkten zusammensetzt. Die Bewertung erfolgt an einem Zinszahlungstermin. Der Kapitalmarktzins zum Bewertungszeitpunkt beträgt 5~\%.
	Die Diskontierung erfolgt mit dem Kapitalmarktzins ohne Spread.

	\begin{align*}
		P_{t+T}^{s}&=\left(B_{t+T}^{s}-B_{t+T}^{0}\right)+100(1+c - s)\left(1+t R_{0}\right)^{-1}\\
		B_{t+T}^{s}&=\frac{0.60}{1.05}+\frac{0.60}{1.05^2}+\frac{0.60}{1.05^3}+\frac{100.60}{1.05^4}=84.40\\
		B_{t+T}^{0}&=\frac{100}{1.05^4}=82.27\\
		P_{t+T}^{s}&= (84.40-82.27) + 100(1+0.056 - 0.006) (1+0.05)^{-1} = 102.13\\
	\end{align*}

	Da eine Zerlegung in einen festen und einen variablen Zahlungsstrom erfolgt, ist zunächst die Festzinsanleihe und die Nullkuponanleihe (erster Term) zu bewerten.

	Die Bewertung des variablen Zahlungsstroms (zweiter Term) vereinfacht sich zu 100, da der variable Zins und der Diskontierungszinssatz identisch sind.
\end{example}

% Mittels eines No-Arbitrage Arguments lässt sich zeigen, dass ein Floater im Fixingtermin immer zu pari notiert. Eine Floating Rate Note lässt sich synthetisch erzeugen durch eine rollierende (revolvierende) Geldanlage in einer Fixed Rate Note. Diese müssen dann denselben Preis haben.

% Achtung!: Dies gilt nur für den Fall, dass der Bonitätsspread dem Bonitätsrisiko entspricht (sem=qm).

% Im hier betrachteten (und einfachsten) Fall gilt für die Zahlung in $t_i$ dass ihre Höhe in $t_{i-1}$ (\textit{reset date}) wird, die Zahlung selbst dagegen in $t_i$ (\textit{payment date}) erfolgt. Dies führt dazu, dass man den Wert der viarablen zinszahlungen und den WERt der variable verzinslcihen Anleihe ohne Zugrundelgung eines Zinsmodells bestimmen kann, sie also unabhängig von dem unterstellten Verhaltne der zinsen über die Zeit sind. DAs soll im Folgenden gezeigt werden. (aus Zinsderivate Branger \& Schlag, ISBN: 3-540-212228-0) Insgesamt ist das Buch aber schlecht, weil es keinen Spread und auch keine Bonitätsveränderung berücksichtigt. 
 
\subsection{Bewertung eines Exchange Traded Funds}
\label{sec:bewertung_eines_exchange_traded_funds}

Bei \glspl{ETF} handelt es sich um eine börsengehandelte Variante des Investmentfonds, die es Anlegern ermöglicht, Portfolios, die einen Index replizieren, zu handeln \autocite[][S.~103]{bodie_investments_2018}. Bei dem zugrundeliegenden Index kann es sich dabei beispielsweise um einen Aktien- oder Anleihenindex handeln, dessen Wertentwicklung abgebildet werden soll.

Die Berechnung des Index setzt Konventionen zur Gewichtung der Anlagen voraus. Ein Überblick über Ansätze zur Gewichtung wird in \textcite[][S.~44~ff.]{bodie_investments_2018} gegeben, wohin gegen sich dieses Fachkonzept auf verwendete Ansätze beschränkt.

Die Software \textit{Anika} bietet jedem Teilnehmer die Möglichkeit, einen ETF zu handeln, der die Wertentwicklung des Index \gls{GMAX} repliziert (siehe nachfolgende Definition \ref{def:gmax}).

\begin{definition}
	\label{def:gmax}
	Der \gls{GMAX} ist ein preisgewichteter (\textit{price weighted}) Aktienindex der Planspielunternehmen, bei dem alle Aktien gleichgewichtet sind.
\end{definition}

Die Berechnung des preisgewichteten Aktienindex \gls{GMAX} erlaubt damit eine einfache und nachvollziehbare Berechnung. Zugleich stellt es einen Ansatz dar, der praktische Bedeutung für Indizes wie dem usprünglichen \textit{Dow Jones Industrial Average} hat, weshalb dieses Verfahren für die Software \textit{Anika} ausgewählt wurde. Dieser Ansatz hat jedoch den Nachteil, dass hoch bewertete Aktien einen größeren Einfluss auf die Indexentwicklung nehmen.

Die Berechnung des \gls{GMAX} wird an Beispiel~\ref{ex:gmax}\footnote{Das Beispiel wurde von \textcite[][S.~44]{bodie_investments_2018} adaptiert.} für ein Spiel mit zwei Planspielunternehmen erläutert.

\begin{example}
\label{ex:gmax}
Legt man einen Aktienkurs für Unternehmen A $25$ ($P_0$) und $30$ in ($P_1$) und für Unternehmen B von $100$ ($P_0$) und $90$ in ($P_1$) zugrunde, dann kann der Indexstand zum Bewertungszeitpunkt wie folgt ermittelt werden:\\
Indexstand \gls{GMAX} ($P_0$) = $\frac{(25 + 100)}{2} = 62.5$\\
Indexstand \gls{GMAX} ($P_1$) = $\frac{(30 + 90)}{2} = 60.$

Die prozentuale Veränderung von $P_0$ auf $P_1$ ergibt damit aus $-\frac{2.5}{62.5} = -4.0~\%$.
\end{example}

 Anhand des \gls{GMAX} wird ein \gls{ETF} konstruiert. Die Ausgestaltung des \gls{ETF} in der Software \textit{Anika} kann der nachfolgenden Definition~\ref{def:etf} entnommen werden.
\begin{definition}
	\label{def:etf}
	Der \gls{ETF} trackt den Index \gls{GMAX} durch vollständige Replizierung. Die \textit{Tracking Difference}\footnote{Der \textit{Tracking Difference} bezeichnet die Renditedifferenz zwischen dem \gls{ETF} und dem abgebildeten Index.} und die \textit{Total Expense Ratio}\footnote{Die \textit{Total Expense Ration} bezeichnet die Gesamtkostenquote des Fonds. Hierunter fallen beispielsweise Kosten zur Erfüllung regulatorischer Anforderungen.} wird eleminiert, wodurch die Wertentwicklung des \gls{ETF} die des \gls{GMAX} 1:1 abbildet.
\end{definition}

Eine Bewertung des \gls{ETF} ist deshalb erst dann möglich, wenn alle im \gls{GMAX} enthaltenen Aktienkurse vorliegen. Aufgrund der in Beispiel~\ref{def:etf} gegebenen Ausgestaltung entspricht der Kurs des \glspl{ETF} dem Preis des \gls{GMAX} in Euro. 
Die Anwendung \textit{Anika} weißt \gls{ETF}-Kurse in \textit{Stücknotierung} aus.

\subsection{Bewertung von Festgeld}
\label{sec:bewertung_von_festgeldern}

Als ein Festgeld wird eine Variante der Termineinlage bezeichnet, dessen Kapital für eine vertraglich vereinbarte Anlagedauer fixiert ist.

Die Ausgestaltung von Festgeldern in der Software \textit{Anika} unterscheidet sich dabei in Teilen von den üblichen am Markt befindlichen Festgeldern. In nachfolgender Definition \ref{def:festgeld} werden deshalb die Konditionen des in \textit{Anika} verwendeten Festgelds dargestellt.

\begin{definition}
	\label{def:festgeld}
	Das Festgeld wird mit dem periodebanhängigen Kapitalmarktzinssatz verzinst und hat eine Laufzeit von einer Periode mit automatischer Prolongation um eine weitere Periode. Teilverfügungen, vollständige Verfügungen und Aufstockungen sind jederzeit durch den Teilnehmer ohne Vorfälligkeitsentschädigung möglich. 
\end{definition}

Weil die Festgelder mit dem jeweiligen Kapitalmarktzinssatz verzinst werden, werden sie mit dem jeweiligen Kapitalsaldo bewertet. 
Festgelder unterliegen als klassische Finanzanlage keinen Kurswertrisiken. Es gilt damit ein theoretischer Kurs von 100~\%.

\section{Ermittlung der Kapitalerträge}
\label{sec:ermittlung_von_wertpapierertraegen}

Nachfolgende Kapitel beschreiben die Ermittlung der Kapitalerträge für die in \textit{Anika} auftretenden Finanzanlagen.

\subsection{Ausschüttungen aus Aktien und ETFs}
\label{sec:ausschuettung_aus_aktie}
Wie bereits in Kapitel \ref{sec:bewertung_von_aktien} beschrieben wurde, beinhaltet der Aktienkurs der Planspielunternehmen bereits die ausgeschütteten Dividendenauszahlungen. Aus diesem Grund soll keine separate Dividendenausschüttung erfolgen. Dies gilt auch für \glspl{ETF}.

\subsection{Zinserträge auf Festgelder}
\label{sec:zinsertraege_auf_festgelder}
Festgelder werden, wie bereits in Kapitel \ref{sec:bewertung_von_festgeldern} beschrieben wurde, mit dem jeweiligen Kapitalmarktzinssatz verzinst. Diese Zinszahlung wird am Ende der jeweiligen Periode auf das Zahlungsmittelkonto des Teilnehmers gutgeschrieben.

\subsection{Zinserträge auf Floating Rate Notes}
\label{sec:zinsertraege_auf_floating_rate_notes}

Die Inhaber von \gls{FRN} erhalten gemäß Definition \ref{def:floater} Zinsausschüttungen.
Wegen der periodischen Zinsfixings und Ausschüttung wird der Zinsbetrag nach nachfolgender Definition \ref{def:zins_floater}\footnote{Diese Definition ist aus \textcite[][S.~52]{veronesi_fixed_2010} adaptiert.} bestimmt.

\begin{definition}
	\label{def:zins_floater}
\begin{align*}
	c_t &= 100 * (r_{t-1} + s)\\
	c_t &= \text{Zinszahlung per $t$}\\
	s &= \text{Spread bei Emission}\\
	r_t &= \text{Referenzzins per $t$}
\end{align*}
$c_t$ wird dem Zahlungsmittelkonto gutgeschrieben.
\end{definition}

Die Berechnung der Zinserträge der \gls{FRN} erfolgt damit bezogen auf den Nennbetrag $(= 100)$ unter Verwendung des Kapitalmarktzinssatzes und des Spreads. Dabei ist der zeitliche Verzug zwischen dem Zinsfixing und dem Termin der Kuponzahlung erkennbar.

\begin{example}
	Eine \gls{FRN} muss bepreist werden. Der Kapitalmarktzins zum zurückliegenden Fixingtermin betrug 2~\%. Der \textit{Spread} bei Emission der \gls{FRN} beträgt 1~\%. 
	
	Die Kuponzahlung $c_t$ ergibt sich wiefolgt:

	$c_t= 100 (0.01 + 0.02) = 3 $ Euro
\end{example}
\clearpage
\printbibliography[title={Literatur}]
\end{document}
