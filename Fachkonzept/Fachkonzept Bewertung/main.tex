\documentclass{article}
\usepackage[titles]{tocloft}
\usepackage{xpatch}
\usepackage{etoolbox}

\usepackage[utf8]{inputenc}
\usepackage[ngerman]{babel}
\usepackage{hyphenat} % manuelle Trennung z. B.  Ma-Risk -> MaRisk
\usepackage[T1]{fontenc}
\usepackage{graphicx}
\usepackage{float}
\usepackage[hyphens]{url}
\usepackage{geometry}
\usepackage{calc}	
\usepackage{fancyhdr}
\usepackage{xcolor}
\usepackage{setspace}
% deaktiviere bunte Boxen, vergebe pdf meta informationen
\usepackage[hidelinks, pdfauthor={Markus Bilz, Christian Fix},
pdftitle={Fachkonzept für die Bewertung von Wertpapieren},
pdfkeywords={Anleihebewertung, Aktienbewertung}
]{hyperref}
\usepackage[raggedright]{titlesec} % Vorgabe für Titelformatierung
\usepackage{caption}
\usepackage{subcaption}
\usepackage{acronym}
\usepackage{float}
\usepackage{enumitem}
\usepackage{booktabs}
\usepackage{tikz}
\usepackage{csquotes} % Behebt Fehler bei Zitaten in Babel
\usepackage{layout} 
\usepackage[all]{nowidow} % vermeidet Witten und Waisen
\usepackage{amsmath}
\usepackage[ngerman, num]{isodate} % wird für Datumsformatierung gebraucht
\usepackage{fncychap}
\usepackage[hang,flushmargin]{footmisc} % Formatierung der Fußnoten
\captionsetup[table]{position=bottom} 
\usepackage{lmodern} 
\usepackage[backend=biber,style=apa,citestyle=apa]{biblatex} % apa-stil für bibliography in biblatex
\DeclareLanguageMapping{german}{german-apa}
\usepackage[toc,section=section]{glossaries} % packet für Glossar ggf. auch für Abkürzungsverzeichnis
\usepackage{booktabs, tabularx, threeparttable}
\usepackage{ragged2e, array}
\usepackage{listings}% Aufzählungen

\usepackage{pgfplots}% pgf plots 
\pgfplotsset{compat=1.15}
\usepgfplotslibrary{statistics}
\usepgfplotslibrary{dateplot}

\geometry{a4paper,
    left=35mm,
    right=25mm,
    top= 25mm, 
    bottom=20mm,
    headsep=12.5mm,
    headheight=12.5mm,
    footskip=12.5mm,
    voffset = 0pt,
    hoffset = 0pt,
}



% Absatzeinstellungen
\setlength{\parindent}{0em}
\setlength{\parskip}{6pt}
\linespread{1.3} % Einfacher Zeilenabstand

% wichtig für Bilder
\graphicspath{{img/}}


% Kopf- und Fußzeilen einbinden
\pagestyle{fancy}
\fancyhf{}
\fancyfoot[R]{\thepage}
\renewcommand{\headrulewidth}{0.4pt} %obere Trennlinie
\renewcommand{\footrulewidth}{0.4pt} %untere Trennlinie
\newcommand{\sectionbreak}{\clearpage} % section auf neuer Seite

% Erhöhe Zeilenabstand bei Tabellen
\renewcommand{\arraystretch}{1.3}

% Einstellungen für underfull und overfull badness
\tolerance 1414
\hbadness 1414
\emergencystretch 1.5em
\widowpenalty=10000
\vfuzz=20pt 
\hfuzz=20pt

\title{Fachkonzept für die Bewertung von Wertpapieren}
\author{Markus Bilz\thanks{markus.bilz@student.dhbw-karlsuhe.de}, Christian Fix\thanks{christian.fix@student.dhbw-karlsuhe.de}}


\begin{document}

\maketitle

\section{Überblick}
Dieses Fachkonzept beschreibt die Bewertung von Wertpapieren in der zu erstellenden Software \textit{Anika}. 
Dabei werden die verwendeten Bewertungsmodelle und notwendige Prämissen eingeführt.

Eine Bewertung von Wertpapieren ist für die

Die Buchung von Transaktionen, die im Zusammenhang mit Wertpapieren stehen, werden als Teils des Fachkonzepts Buchung beschrieben.
% Kauf und Verkauf welcher Kurs?

\section{Prämissen}
Für die Bewertung von Wertpapieren 





\section{Zeitpunkt und Durchführung der Bewertung}
\label{sec:zeitpunkt_und_durchfuehrung_der_bewertung}

\begin{figure}[htb]
	\centering
	\begin{tikzpicture} 
		\draw[thick, ->] (0,0) -- (12cm,0);
		\foreach \x in {2,4,6,8,10}
		\draw (\x cm,3pt) -- (\x cm,-3pt);
		\draw[thick] (4,0) node[below=3pt,thick] {$t_0$} node[above=3pt] {};
		\draw[thick] (6,0) node[below=3pt,thick] {$\dots$} node[above=3pt] {};
		\draw[thick] (8,0) node[below=3pt, thick] {$t_h$} node[above=3pt] {};
		\draw[thick] (10,0) node[below=3pt] {$T$} node[above=3pt] {};
		\draw [black, thick ,decorate,decoration={brace,amplitude=5pt}] (4,0.5)  -- (8,0.5) 
			   node [black,midway,above=4pt,font=\scriptsize] {$h$};
		\draw [ black, thick,decorate,decoration={brace,amplitude=5pt},yshift=-11pt] (10,-0.5) -- (8,-0.5)
			   node [black,midway,below=4pt,font=\scriptsize] {$T-t_h$};
		\end{tikzpicture}
	\caption{Optionsbewertung in der historischen Simulation}
	\label{img:zeitstrahl_var}
	(Eigene Darstellung)
\end{figure}

\section{Bewertung von Aktien}
Die Unternehmen des Planspiels firmieren als Aktiengesellschaft.
Ein rechnerischer Aktienkurs wird daher durch die Anwendung \textit{TOPSIM} auf Grundlage des Eigenkapitals und des Jahresüberschusses der vergangenen Periode nebst anderen Einflussfaktoren berechnet (\dots). 
Eine Gewichtung der Faktoren mit Einfluss auf den Aktienkurs ist in der Anwendung \textit{TOPSIM} konfigurierbar. 

Aktienkurse der Planspielunternehmen werden \textit{ex post} je Planspielunternehmen und Periode ermittelt. Die durch die Software \textit{TOPSIM} ermittelten Aktienkurse sind um Dividendenausschüttungen bereingt. Kapitalmaßnahmen mit Einfluss auf den Aktienkurs erfolgen darüber hinaus nicht, womit eine unmittelbare Verwendung in der Software \textit{Anika} möglich ist. Nachfolgend wird beschrieben, wie eine Erfassung der Aktienkurse erfolgen soll.

\subsection{Erfassung von Aktienkursen}


Gemäß Kapitel \ref{sec:zeitpunkt_und_durchfuehrung_der_bewertung} ist der Aktienkurs der Vorperiode der Bewertungskurs der Folgeperiode.
Zugleich ist ein Kauf und Verkauf zu diesem Kurs möglich -- bleiben Ordergebühren und eine Geld-/ und Briefspanne außer Acht. 

\subsection{Speicherung von Aktienkursen}

\subsection{Bewertung von Portfoliopositionen}





\section{Bewertung von Anleihen}

Anleihen sind kein Bestandteil der Anwendung \textit{TOPSIM}, insofern ist eine Bewertung durch die Anwendung \textit{Anika} notwendig.
Hierbei ist insbesondere eine Bewertung der An

Für die Implementierung wird eine Laufzeit von zehn Jahren, beginnend in Periode Null angenommen.

\section{Bewertung des Aktien Exchange Traded Funds}

Für die Bewertung des ETFs wird dabei eine vollständige Replizierung des Indizes unterstellt.
Der \textit{Tracking Error} und die \textit{Total Expense Ration} werden dabei mit Null angenommen.
Dies bedeutet, dass der ETF der Entwicklung des zugrundeliegenden Indizes 1:1 folgt.

Für die rechnerische Bewertung ist deshalb die Berechnung des \textit{GMAX} Indizes durch die Anwendung erforderlich.

\subsection{Berechnung des \textit{GMAX} Index}

\subsection{Speicherung von ETF Kursen}

\subsection{Bewertung von Portfoliopositionen}
Siehe \dots

% https://en.wikipedia.org/wiki/Stock_market_index
% \section{Ausblick}
% Das Grundsätzlich ist auch eine Bewert
% Weiterhin ist . Die propagierten Modelle sind dabei geeignet, um eine 

\end{document}
