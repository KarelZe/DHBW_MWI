\documentclass[12pt, a4paper]{article}
\usepackage[titles]{tocloft}
\usepackage{xpatch}
\usepackage{etoolbox}

\usepackage[utf8]{inputenc}
\usepackage[ngerman]{babel}
\usepackage{hyphenat} % manuelle Trennung z. B.  Ma-Risk -> MaRisk
\usepackage[T1]{fontenc}
\usepackage{graphicx}
\usepackage{float}
\usepackage[hyphens]{url}
\usepackage{geometry}
\usepackage{calc}	
\usepackage{fancyhdr}
\usepackage{xcolor}
\usepackage{setspace}
% deaktiviere bunte Boxen, vergebe pdf meta informationen
\usepackage[hidelinks, pdfauthor={Markus Bilz, Christian Fix},
pdftitle={Fachkonzept für die Bewertung von Wertpapieren},
pdfkeywords={Anleihebewertung, Aktienbewertung}
]{hyperref}
% \usepackage[raggedright]{titlesec} % Vorgabe für Titelformatierung
\usepackage{caption}
\usepackage{subcaption}
\usepackage{acronym}
\usepackage{float}
\usepackage{enumitem}
\usepackage{booktabs}
\usepackage{tikz}
\usepackage{csquotes} % Behebt Fehler bei Zitaten in Babel
\usepackage{layout} 
\usepackage[all]{nowidow} % vermeidet Witten und Waisen
\usepackage{amsmath}
\usepackage[ngerman, num]{isodate} % wird für Datumsformatierung gebraucht
\usepackage{fncychap}
\usepackage[hang,flushmargin]{footmisc} % Formatierung der Fußnoten
\usepackage{lmodern} 
\usepackage[backend=biber,style=apa,citestyle=apa]{biblatex} % apa-stil für bibliography in biblatex
\DeclareLanguageMapping{german}{german-apa}
\usepackage[toc,section=section]{glossaries} % packet für Glossar ggf. auch für Abkürzungsverzeichnis
\usepackage{booktabs, tabularx, threeparttable}
\usepackage{ragged2e, array}
\usepackage{listings}% Aufzählungen

\usepackage{pgfplots}% pgf plots 
\pgfplotsset{compat=1.15}
\usepgfplotslibrary{statistics}
\usepgfplotslibrary{dateplot}

\geometry{a4paper,
    left=35mm,
    right=25mm,
    top= 25mm, 
    bottom=20mm,
    headsep=12.5mm,
    headheight=12.5mm,
    footskip=12.5mm,
    voffset = 0pt,
    hoffset = 0pt,
}



% Absatzeinstellungen
\setlength{\parindent}{0em}
\setlength{\parskip}{6pt}
\linespread{1.3} % Einfacher Zeilenabstand

% wichtig für Bilder
\graphicspath{{img/}}


% Kopf- und Fußzeilen einbinden
\pagestyle{fancy}
\fancyhf{}
\fancyfoot[R]{\thepage}
\renewcommand{\headrulewidth}{0.4pt} %obere Trennlinie
\renewcommand{\footrulewidth}{0.4pt} %untere Trennlinie
\newcommand{\sectionbreak}{\clearpage} % section auf neuer Seite

% Erhöhe Zeilenabstand bei Tabellen
\renewcommand{\arraystretch}{1.3}

% Einstellungen für underfull und overfull badness
\tolerance 1414
\hbadness 1414
\emergencystretch 1.5em
\widowpenalty=10000
\vfuzz=20pt 
\hfuzz=20pt

\title{Fachkonzept für die Bewertung von Wertpapieren}
\author{Markus Bilz\thanks{markus.bilz@student.dhbw-karlsuhe.de}, Christian Fix\thanks{christian.fix@student.dhbw-karlsuhe.de}}

\addbibresource{bib.bib}

\begin{document}

\maketitle

\section{Überblick*}
Dieses Fachkonzept beschreibt die Bewertung von Wertpapieren in der zu erstellenden Software \textit{Anika}. 
Dabei werden die verwendeten Bewertungsmodelle und notwendige Prämissen eingeführt.

Da die Software \textit{Anika} kein dediziertes Handelssystem implementiert, das Handelsaktivitäten einzelner Marktteilnehmer beispielsweise Kauforders der Planspielteilnehmer in der Kursbildung von Wertpapieren berücksichtigt, werden anwendungsweit rechnerische Kurswerte verwendet. 

Die Bewertung als auch der Handel von Finanzanlagen erfolgt deshalb zu einem rechnerischen Kurs, der auf Grundlage finanzmathematischer Modelle ermittelt wird. Insofern ist die Modellierung des rechnerischen Werts einer Finanzanlage eine zentrale Funktionalität der Software.

Die Software \textit{Anika} unterstützt dabei die Bepreisung folgender Finanzanlagen:
\begin{enumerate}
	\item Aktien
	\item Floating Rate Notes
	\item Exchange Traded Funds
	\item Tagesgeld
\end{enumerate}

Nachfolgende Kapitel gehen detailliert auf die Bewertung obiger Anlagearten ein. Zunächst wird jedoch der Zeitpunkt und das Vorgehen hinsichtlich der Bewertung von Finanzanlagen thematisiert.

\section{Zeitpunkt und Durchführung der Bewertung}
\label{sec:zeitpunkt_und_durchfuehrung_der_bewertung}

\begin{figure}[htb]
	\centering
	\begin{tikzpicture} 
		\draw[thick, ->] (0,0) -- (12cm,0);
		\foreach \x in {2,4,6,8,10}
		\draw (\x cm,3pt) -- (\x cm,-3pt);
		\draw[thick] (4,0) node[below=3pt,thick] {$t_0$} node[above=3pt] {};
		\draw[thick] (6,0) node[below=3pt,thick] {$\dots$} node[above=3pt] {};
		\draw[thick] (8,0) node[below=3pt, thick] {$t_h$} node[above=3pt] {};
		\draw[thick] (10,0) node[below=3pt] {$T$} node[above=3pt] {};
		\draw [black, thick ,decorate,decoration={brace,amplitude=5pt}] (4,0.5)  -- (8,0.5) 
			   node [black,midway,above=4pt,font=\scriptsize] {$h$};
		\draw [ black, thick,decorate,decoration={brace,amplitude=5pt},yshift=-11pt] (10,-0.5) -- (8,-0.5)
			   node [black,midway,below=4pt,font=\scriptsize] {$T-t_h$};
		\end{tikzpicture}
	\caption{Optionsbewertung in der historischen Simulation}
	\label{img:zeitstrahl_var}
	(Eigene Darstellung)
\end{figure}

\section{Bewertung von Aktien}
Die Unternehmen des Planspiels firmieren als Aktiengesellschaft.
Ein rechnerischer Aktienkurs wird daher durch die Anwendung \textit{TOPSIM} auf Grundlage des Eigenkapitals und des Jahresüberschusses der vergangenen Periode nebst anderen Einflussfaktoren berechnet (\dots). 
Eine Gewichtung der Faktoren mit Einfluss auf den Aktienkurs ist in der Anwendung \textit{TOPSIM} konfigurierbar. 

Aktienkurse der Planspielunternehmen werden \textit{ex post} je Planspielunternehmen und Periode ermittelt. Die durch die Software \textit{TOPSIM} ermittelten Aktienkurse sind um Dividendenausschüttungen bereingt. Kapitalmaßnahmen mit Einfluss auf den Aktienkurs erfolgen darüber hinaus nicht, womit eine unmittelbare Verwendung in der Software \textit{Anika} möglich ist. Nachfolgend wird beschrieben, wie eine Erfassung der Aktienkurse erfolgen soll.

\subsection{Prämisse}

\subsection{Erfassung von Aktienkursen}


Gemäß Kapitel \ref{sec:zeitpunkt_und_durchfuehrung_der_bewertung} ist der Aktienkurs der Vorperiode der Bewertungskurs der Folgeperiode.
Zugleich ist ein Kauf und Verkauf zu diesem Kurs möglich -- bleiben Ordergebühren und eine Geld-/ und Briefspanne außer Acht. 

\subsection{Speicherung von Aktienkursen}

\subsection{Bewertung von Portfoliopositionen}





\section{Bewertung von Anleihen}

Anleihen sind kein Bestandteil der Anwendung \textit{TOPSIM}, insofern ist eine Bewertung durch die Anwendung \textit{Anika} notwendig.
Hierbei ist insbesondere eine Bewertung der An

Für die Implementierung wird eine Laufzeit von zehn Jahren, beginnend in Periode Null angenommen.

Da Zeitpunkt der Zinszahlung und Kauf- / und Zeitpunkt der Anleihe übereinstimmen, sind keine Stückzinsen zu berücksichtigen.

Was sind Floating Rate Notes? Siehe \autocite[][]{fabozzi_fixed_2007}.

Die Bewertung orientert sich an \autocite[][]{veronesi_fixed_2010}.

Der Wert eines Floaters mit einem Spread von $s=0$ entspricht der der Wert des Floater dem Nennwert des Floaters ohne Zinskupons \autocite[][S.~52~f.]{veronesi_fixed_2010}. Dies ist auf \glqq Zahlungsstromeffekte\grqq~und \glqq Diskontierungseffekte\grqq~zurückzuführen, die sich gegenseitig ausgleichen \autocite[][S.~54]{veronesi_fixed_2010}. Höhere Zinszahlungen -- resultierend aus einer Zinserhöhung -- werden durch einen höheren Diskontierungssätze kompensiert.

Es handelt sich dabei um einen Spezialfall, der die Anforderungen von \textit{Anika} nicht vollständig abdeckt, da zwar eine Bepreisung zum Zinszahlungstermin erfolgt, der Spread aber auch andere Werte annehmen kann.

Für einen Floater mit einem Spread lässt sich der Cashflow in eine fixe und eine veränderliche Komponente aufspalten.

Damit teilt sich die Bewertung in die Bewetung einer Floating Rate Note mit einem Spread von Null und mehreren festen Zinszahlungen in Höhe des Spreads auf.


\begin{align}
	V(t)=\sum_{i=1}^{n} F_{i} \tau_{i} P e^{-\left(r_{i}+s\right) T_{i}}+P e^{-\left(r_{n}+s\right) T_{n}}
\end{align}

% https://www.finpricing.com/lib/FiFrn.pdf

Ei

\subsection{Berechnung von Forward Rates}

\section{Bewertung des Aktien Exchange Traded Funds}

Für die Bewertung des ETFs wird dabei eine vollständige Replizierung des Indizes unterstellt.
Der \textit{Tracking Error} und die \textit{Total Expense Ration} werden dabei mit Null angenommen.
Dies bedeutet, dass der ETF der Entwicklung des zugrundeliegenden Indizes 1:1 folgt.

Für die rechnerische Bewertung ist deshalb die Berechnung des \textit{GMAX} Indizes durch die Anwendung erforderlich.

\subsection{Berechnung des {GMAX} Index}

\subsection{Speicherung von {ETF} Kursen}

\subsection{Bewertung von Portfoliopositionen}
Siehe \dots

% https://en.wikipedia.org/wiki/Stock_market_index
% \section{Ausblick}
% Das Grundsätzlich ist auch eine Bewert
% Weiterhin ist . Die propagierten Modelle sind dabei geeignet, um eine 

% Kurz beschreiben

\section{Bewertung von Festgeldern}

% Verfügung über Festgelder @ Raphaels Fachkonzept. Ist es nicht sinnvoller, das Festgeld als Tagesgeld zu bezeichnen, wegen der jederzeitigen Verfügung? -> Auf Konsistenz zu Code achten. Faktisch wird aber verzinst zu Periodenabhängigem Kapitalmarktzins.


Festgelder werden zum Nennbetrag bewertet.

% TODO: Insolvenzfall für alle Assets betrachten.

\printbibliography[title={Literatur}]
\end{document}
