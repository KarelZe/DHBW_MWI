\documentclass[12pt, a4paper]{article}
\usepackage[titles]{tocloft}
\usepackage{xpatch}
\usepackage{etoolbox}

\usepackage[utf8]{inputenc}
\usepackage[ngerman]{babel}
\usepackage{hyphenat} % manuelle Trennung z. B.  Ma-Risk -> MaRisk
\usepackage[T1]{fontenc}
\usepackage{graphicx}
\usepackage[hyphens]{url}
\usepackage{geometry}
\usepackage{calc}	
\usepackage{fancyhdr}
\usepackage{xcolor}
\usepackage{setspace}
% deaktiviere bunte Boxen, vergebe pdf meta informationen
\usepackage[hidelinks, pdfauthor={Markus Bilz, Christian Fix},
pdftitle={Fachkonzept für die Bewertung von Wertpapieren},
pdfkeywords={Anleihebewertung, Aktienbewertung}
]{hyperref}
% \usepackage[raggedright]{titlesec} % Vorgabe für Titelformatierung
\usepackage{caption}
\usepackage{subcaption}
\usepackage{acronym}
\usepackage{float}
\usepackage{enumitem}
\usepackage{booktabs}
\usepackage{tikz}
\usepackage{csquotes} % Behebt Fehler bei Zitaten in Babel
\usepackage{layout} 
\usepackage[all]{nowidow} % vermeidet Witten und Waisen
\usepackage{amsmath}
\usepackage[ngerman, num]{isodate} % wird für Datumsformatierung gebraucht
\usepackage{fncychap}
\usepackage[hang,flushmargin]{footmisc} % Formatierung der Fußnoten
\usepackage{lmodern} 
\usepackage[backend=biber,style=apa,citestyle=apa]{biblatex} % apa-stil für bibliography in biblatex
\DeclareLanguageMapping{german}{german-apa}
\usepackage[toc,section=section]{glossaries} % packet für Glossar ggf. auch für Abkürzungsverzeichnis
\usepackage{booktabs, tabularx, threeparttable}
\usepackage{ragged2e, array}
\usepackage{listings}% Aufzählungen

\usepackage{pgfplots}% pgf plots 
\pgfplotsset{compat=1.15}
\usepgfplotslibrary{statistics}
\usepgfplotslibrary{dateplot}

\geometry{a4paper,
    left=35mm,
    right=25mm,
    top= 25mm, 
    bottom=20mm,
    headsep=12.5mm,
    headheight=12.5mm,
    footskip=12.5mm,
    voffset = 0pt,
    hoffset = 0pt,
}

% Absatzeinstellungen
\setlength{\parindent}{0em}
\setlength{\parskip}{6pt}
\linespread{1.3} % Einfacher Zeilenabstand

% wichtig für Bilder
\graphicspath{{img/}}


% Kopf- und Fußzeilen einbinden
\pagestyle{fancy}
\fancyhf{}
\fancyfoot[R]{\thepage}
\renewcommand{\headrulewidth}{0.4pt} %obere Trennlinie
\renewcommand{\footrulewidth}{0.4pt} %untere Trennlinie
\newcommand{\sectionbreak}{\clearpage} % section auf neuer Seite

% Erhöhe Zeilenabstand bei Tabellen
\renewcommand{\arraystretch}{1.3}

% Einstellungen für underfull und overfull badness
\tolerance 1414
\hbadness 1414
\emergencystretch 1.5em
\widowpenalty=10000
\vfuzz=20pt 
\hfuzz=20pt

\title{Fachkonzept für die Bewertung von Finanzanlagen und die Verbuchung von Kapitalerträgen}
\author{Markus Bilz\thanks{markus.bilz@student.dhbw-karlsuhe.de}, Christian Fix\thanks{christian.fix@student.dhbw-karlsuhe.de}}

\addbibresource{bib.bib}

\newacronym{ETF}{ETF}{Exchange Traded Fund}
\newacronym{FRN}{FRN}{Floating Rate Note}

\makeatletter
\let\@fnsymbol\@arabic
\makeatother

\begin{document}

\maketitle

\section{Überblick*}
Im Rahmen des MWI-Projektes soll die bestehende Planspielsoftware in der DHBW Karlsruhe um eine zusätzliche autarke Wertpapierkomponente mit dem Namen \textit{Anika} erweitert werden.
Um sicherzustellen, dass diese Wertpapiere möglichst realistisch und fair bewertet werden können, beschreibt dieses Fachkonzept, wie deren Bewertung in dieser Software durchgeführt werden soll.

Die Software \textit{Anika} unterstützt dabei den Handel folgender Finanzanlagen:
\begin{enumerate}
	\item Aktien
	\item \gls{FRN}
	\item \gls{ETF}
	\item Festgeld
\end{enumerate}

Weil \textit{Anika} kein dediziertes Handelssystem implementiert, das die Handelsaktivitäten einzelner Marktteilnehmer in der Kursbildung von Wertpapieren berücksichtigt, werden anwendungsweit ausschließlich rechnerische Kurswerte verwendet. 
Deshalb erfolgt sowohl die Bewertung als auch der Handel von Finanzanlagen zu einem rechnerischen Kurs, der auf Grundlage finanzmathematischer Modelle ermittelt wird.
Insofern ist die Modellierung des rechnerischen Werts einer Finanzanlage eine zentrale Funktionalität der Software.

%Nachfolgende Kapitel gehen detailliert auf die Bewertung obiger Anlagearten ein. Zunächst wird jedoch der Zeitpunkt und das Vorgehen hinsichtlich der Bewertung von Finanzanlagen thematisiert.

\section{Anlage und Pflege von Finanzanlagen $\ast$}
\label{sec:anlage_und_pflege_der_wertpapiere}

Bevor eine Bewertung der Wertpapiere erfolgen kann, müssen die handelbaren Wertpapiere einschließlich der für die Bewertung notwendigen Geschäftsdaten angelegt sein. 
Im Folgenden wird beschrieben, wie die Finanzanlagen angelegt und gepflegt werden:
\begin{itemize}
	\item Der \gls{ETF} und das Festgeld wird automatisch bei der Anlage eines Spiels erstellt.
	\item Aktien und \glspl{FRN} können von den Planspielunternehmen emittiert werden, indem der Spielleiter dies bei der Anlage des Spiels manuell durchführt. Die Software ermöglicht dabei, dass je Planspielunternehmen $\geq 0$ Aktien und \glspl{FRN} emittiert werden können. 
\end{itemize}

Die Pflege der für die Bewertung notwendigen Daten wie beispielsweise die Erfassung des Kapitalmarktzinssatzes, des Risikoaufschlages oder des Aktienkurses erfolgt dabei einmalig vor dem Start einer Planspielperiode durch den Spielleiter.

\section{Zeitpunkt der Bewertung und Verbuchung}
\label{sec:zeitpunkt_und_durchfuehrung_der_bewertung_buchung}

Dieses Kapitel beschreibt den Zeitpunkt der Bewertung und der Verbuchung von Kapitelerträgen.

Die Software \textit{TOPSIM} unterteilt ein Planspiel in $n$ Perioden $P$.
Eine feingranulare Unterteilung einer Periode ist nicht möglich, weshalb die Dauer einer Periode mit einer Zeiteinheit angenommen wird.
Infolge entspricht der Periodenbeginn von $P_1$ dem Ultimo der Vorperiode $P_0$.

Demnach ergibt sich folgender Zusammmenhang:

\begin{figure}[htb]
	\centering
	\begin{tikzpicture} 
		\draw[thick, ->] (0,0) -- (12cm,0);
		\foreach \x in {0,4,8}
		\draw (\x cm,3pt) -- (\x cm,-3pt);
		\draw[thick] (4,0) node[below=3pt,thick] {Ultimo $P_0$} node[above=3pt] {};
		\draw[thick] (8,0) node[below=3pt, thick] {Ultimo $P_1$} node[above=3pt] {};
		\draw [thick ] (4,0.5) node [above=4pt,font=\scriptsize, align=left] {
		Verbuchung $P_0$\\ \parindent=1em \indent Bewertung $P_1$\\ \parindent=2em \indent Kauf/Verkauf $P_1$};
		\draw [ black, thick,decorate,decoration={brace,amplitude=5pt},yshift=-11pt] (8,-0.5) -- (4,-0.5)
			   node [black,midway,below=4pt,font=\scriptsize] {Dauer Verzinsung};
		\draw [thick ] (8,0.5) node [above=4pt,font=\scriptsize, align=left] {Verbuchung $P_1$\\ \parindent=1em \indent Bewertung $P_2$\\ \parindent=2em \indent Kauf/Verkauf $P_2$};
	\end{tikzpicture}
	\caption{Bewertungs-/ Buchungszeitpunkt}
	\label{img:zeitstrahl_pewertung}
	(Eigene Darstellung)
\end{figure}

Damit lässt sich zusammenfassen, dass Aktienkurse und rechnerische Anleihekurse -- festgestellt am Ultimo der Vorperiode -- die für die Folgeperiode relevanten Kurse für die Bewertung und den Handel darstellen.

Die Bewertung der Finanzanlagen kann in einer beliebigen Reihenfolge efolgen.
Ausschließlich für die Bewertung des \glspl{ETF} bestehen temporale Abhängigkeiten zu anderen Anlagen. Kapitel \ref{sec:bewertung_eines_exchange_traded_funds} thematisiert dies detailliert.

Ein Kauf und Verkauf von Finanzanlagen ist erst möglich, nachdem die Finanzanlagen bewertet wurden.

Die Verbuchung der Kapitalerträge erfolgt jeweils am Ultimo der Periode nach Durchführung aller Kauf- und Verkaufbuchungen.
Die Dauer der Verzinsung beträgt damit ein Tag.

\section{Bewertung von Finanzanlagen}
\label{sec:bewertung_von_finanzanlagen}
Im Folgenden wird beschrieben, wie die Finanzanlagen in der Software \textit{Anika} bewertet werden sollen.

\subsection{Bewertung von Aktien $\ast$}
\label{sec:bewertung_von_aktien}
Die Planspielunternehmen firmieren als Aktiengesellschaft, deren Aktien von den Teilnehmern gehandelt werden können.
Auf der Basis einiger Einflussfaktoren berechnet die Planspielsoftware \textit{TOPSIM} auf der Basis einiger Einflussfaktoren wie beispielsweise dem Eigenkapital oder dem Jahresüberschuss der vergangenen Periode einen rechnerischer Aktienkurs.
Die Methodik, wie sich dieser Aktienkurs berechnet wird, ist dabei grundsätzlich konfigurierbar.

Die Aktienkurse der Planspielunternehmen werden \textit{ex post} Periode ermittelt und dem Seminarleiter in einer Übersicht dargestellt. Der darauf abgebildete Kurs beinhaltet die vergangenen Dividendenauszahlungen, wodurch der Inhaber der Aktie keine Dividende ausgezahlt bekommt, statdessen erhöht sich der Kurs der Aktie.

Gemäß Kapitel \ref{sec:zeitpunkt_und_durchfuehrung_der_bewertung_buchung} ist der Aktienkurs der Vorperiode der Bewertungskurs der Folgeperiode.
Bei dem Handel mit Aktien wird eine vom Seminarleiter eingestellte Ordergebührt fällig. Eine in der Realität oft auftretende Brief-Geld-Spanne existiert hingegen nicht.

% TODO: Ausformulieren. Pflege von Kursen siehe Kapitel 2. Ordergebühren in %. Zuordnung sinnvoll zu Handel von Wertpapieren?

\subsection{Bewertung von Floating Rate Notes $\ast$}
\label{sec:bewertung_von_floating_rate_notes}
\glspl{FRN} sind Anleihen mit einem über die Laufzeit veränderlichen Zinskupon \autocite[][373]{fabozzi_handbook_2005}. Der Zinskupon setzt sich dabei aus einem aus einem Referenzzins und einen von der Bonität des Emittenten abhängigen Zinsaufschlages zusammen \autocite[][374]{fabozzi_handbook_2005}.

Anleihen sind kein Bestandteil der Anwendung \textit{TOPSIM}, insofern ist eine Bewertung durch die Anwendung \textit{Anika} notwendig.
Hierbei ist insbesondere eine Bewertung der An

Für die Implementierung wird eine Laufzeit von zehn Perioden, beginnend in Periode Null angenommen.
Die \gls{FRN} verfügt über keinen Cap, Floor oder Collar, der die Höhe des Zinskupons beschränkt.
Kündigungsrechte des Emittenten / Gläubigers bestehen nicht.

Da Zeitpunkt der Zinszahlung und Kauf- / und Zeitpunkt der Anleihe übereinstimmen, sind keine Stückzinsen zu berücksichtigen.

Die Bewertung orientert sich an \autocite[][]{veronesi_fixed_2010}.

Alternativ ist auch \autocite[][]{fabozzi_handbook_2005} möglich.

Der Wert einer \gls{FRN} mit einem Spread von $s=0$ entspricht der der Wert der \gls{FRN} dem Nennwert der \gls{FRN} ohne Zinskupons \autocite[][S.~52~f.]{veronesi_fixed_2010}. Dies ist auf \glqq Zahlungsstromeffekte\grqq~und \glqq Diskontierungseffekte\grqq~zurückzuführen, die sich gegenseitig ausgleichen \autocite[][S.~54]{veronesi_fixed_2010}. Höhere Zinszahlungen -- resultierend aus einer Zinserhöhung -- werden durch einen höheren Diskontierungssätze kompensiert.

Es handelt sich dabei um einen Spezialfall, der die Anforderungen von \textit{Anika} nicht vollständig abdeckt, da zwar eine Bepreisung zum Zinszahlungstermin erfolgt, der Spread aber auch andere Werte annehmen kann.

Für einer \gls{FRN} mit einem Spread lässt sich der Cashflow in eine fixe und eine veränderliche Komponente aufspalten.

Damit teilt sich die Bewertung in die Bewetung einer \gls{FRN} mit einem Spread von Null und mehreren festen Zinszahlungen in Höhe des Spreads auf.

% https://www.finpricing.com/lib/FiFrn.pdf

\subsection{Bewertung eines Exchange Traded Funds $\dagger$}
\label{sec:bewertung_eines_exchange_traded_funds}
Bei \glspl{ETF} handelt es sich um eine börsengehandelte Variante des Investmentfonds, die es Anlegern ermöglicht, Portfolios, die einen Index replizieren, zu handeln \autocite[][S.~103]{bodie_investments_2018}.

Die Software \textit{Anika} soll jedem Teilnehmer die Möglichkeit bieten, einen ETF zu handeln, der die Wertentwicklung des GMAX\footnote{Der Performanceindex GMAX bildet die gleichgewichteten Aktienkurse aller Planspielunternehmen ab.} abbildet.
Deshalb ist eine Bewertung des ETF erst dann möglich, wenn alle im GMAX enthaltenenen Aktienkurse vorliegen.

Für die Bewertung eines \glspl{ETF} wird dabei eine vollständige Replizierung des Indizes unterstellt. Außerdem wird angenommen, dass die \textit{Tracking Difference}\footnote{Der \textit{Tracking Difference} bezeichnet die Renditedifferenz zwischen dem \gls{ETF} und dem abgebildeten Index.} und die \textit{Total Expense Ratio}\footnote{Die \textit{Total Expense Ration} bezeichnet die Gesamtkostenquote des Fonds. Hierunter fallen beispielsweise Kosten zur Erfüllung regulatorischer Anforderungen.} werden dabei nicht berücksichtigt.
Dies führt dazu, dass der \gls{ETF} der Entwicklung des zugrundeliegenden Indizes 1:1 folgt.

\subsection{Bewertung von Festgeld $\ast$}
\label{sec:bewertung_von_festgeldern}

Als ein Festgeld wird eine Variante der Termineinlage bezeichnet, dessen Kapital für eine vertraglich vereinbarte Anlagedauer fixiert ist.

Die Ausgestaltung von Festgeldern in der Software \textit{Anika} unterscheidet sich dabei in Teilen von den üblichen am Markt befindlichen Festgeldern. Im Folgenden werden deshalb die Konditionen des in \textit{Anika} verwendeten Festgeldes dargestellt:
\begin{itemize}
	\item Die Anlagedauer beträgt eine Periode.
	\item Die Verzinsung entspricht dem Kapitalmarktzins der jeweiligen Periode.
	\item Es erfolgt eine automatische Prolongation des Festgelds um eine Periode.
	\item Eine jederzeitige Verfügung den Teilnehmer ist im vollen Umfang oder teilweise ohne Vorfälligkeitsentschädigung möglich.
	\item Aufstockungen des Festgeld sind jederzeit möglich.
\end{itemize}

Weil die Festgelder mit dem jeweiligen Kapitalmarktzinssatz verzinst werden, werden sie mit dem jeweiligen Kapitalsaldo bewertet, da sie dadurch effektiv einer risikolosen Floating Rate Note gleichzusetzen sind. Im Gegensatz dazu fällt jedoch keine Ordergebühr an, da es kein Wertpapier ist.

\section{Ermittlung der Kapitalerträge $\dagger$}
\label{sec:ermittlung_von_wertpapierertraegen}

\subsection{Ausschüttungen aus Aktien und ETFs $\ast$}
\label{sec:ausschuettung_aus_aktie}
% TODO: Gibt es eine klarere Bezeichnung als um Dividendenausschüttungen bereinigt?
Wie bereits in Kapitel \ref{sec:bewertung_von_aktien} beschrieben wurde, beinhaltet der Aktienkurs der Planspielunternehmen die bereits ausgeschütteten Dividendenauszahlungen. Aus diesem Grund soll keine zusätzliche Dividendenausschüttung erfolgen. Dies gilt auch für ETFs.

\subsection{Zinserträge auf Festgelder $\ast$}
\label{sec:zinsertraege_auf_festgelder}
Festgelder werden wie bereits in Kapitel \ref{sec:bewertung_von_festgeldern} beschrieben wurde, mit dem jeweiligen Kapitalmarktzinssatz verzinst. Diese Zinszahlung wird am Ende der jeweiligen Periode auf das Zahlungsmittelkonto des Teilnehmers gutgeschrieben.

\subsection{Zinserträge auf Floating Rate Notes $\ast$}
\label{sec:zinsertraege_auf_floating_rate_notes}

\printbibliography[title={Literatur}]
\end{document}
